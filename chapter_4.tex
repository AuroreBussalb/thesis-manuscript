\chapter{Analyse de la distribution d'un marqueur de l'attention au sein d'une population d'enfants TDAH}

\section*{Introduction}
Diminuer le \gls{tbr} est un protocole d'entrainement de \gls{nfb} couramment utilisé pour traiter le \gls{tdah} chez les enfants 
\citep{Arnold2014, Deilami2016, Gevensleben2009, VanDongen2013}. Cependant, ce protocole ne serait peut-être pas adapté à tous les enfants \gls{tdah} si 
on se base sur le phénotype de leur \gls{eeg}. En effet, il a été avancé qu'il existerait un groupe d'enfants \gls{tdah} présentant un \gls{tbr} élevé 
\citep{Zhang2017, Clarke2011}, ainsi ces enfants bénéficiraient peut-être davantage d'un protocole diminuant leur \gls{tbr} que les autres. 

Quelques études ont proposé de personnaliser le protocole d'entrainement par \gls{nfb} \citep{Bazanova2018, Escolano2014}, mais trop peu pour en déterminer
l'impact sur l'efficacité du \gls{nfb} par la \gls{saob}. L'analyse présentée dans ce chapitre n'a pas pour but d'évaluer directement l'efficacité de la 
personnalisation des protocoles de \gls{nfb} mais sa pertinence. Pour ce faire, la distribution des valeurs de \gls{tbr} chez les enfants \gls{tdah} est 
étudiée à l'aide de différentes méthodes de partitionnement : si un groupe d'enfants avec un \gls{tbr} élevé est mis en évidence, la personnalisation 
des protocoles est envisageable. 

\section{Population étudiée}

Les données utilisées dans cette analyse proviennent de trois bases de données différentes :
\begin{itemize}
\item NEWROFEED (NCT02778360, Mensia Technologies, France, ClinicalTrials.gov, \citet{Bioulac2019}),
\item \gls{cmi-mipdb} \citep{Langer2017, Langer2017b},
\item \gls{cmi-hbn} \citep{Alexander2017, Alexander2017b}.
\end{itemize}
Pour chacune de ces bases de données, un consentement éclairé écrit a été obtenu de tous les participants ou de leurs responsables légaux. Tous les enregistrements
ont été effectués dans un environnement contrôlé avec les yeux ouverts (\gls{eo} en anglais) et au repos (c'est à dire que le sujet n'effectue aucune tâche) 
pendant une minute sous la supervision d'un clinicien ou d'un chercheur. 

La description de l'ensemble des données est disponible dans la Table~\ref{Table:tbr_datasets_description}.

\begin{table}[h!]
  \centering
  \caption{Informations sur les données utilisées. Le critère d'inclusion pour chaque base de données est précisé, ainsi que le nombre de sujets satisfaisant
	chaque critère entre parenthèses. Le nombre total de sujets inclus par base de données est donné à la dernère ligne.}
  \fontsize{9}{11}\selectfont
\begin{tabular}{ ccccc }
\toprule
Base de données & NEWROFEED & CMI-MIPDB & \multicolumn{2}{ c }{CMI-HBN} \\
\midrule
\shortstack{ Description \\ de la population } & \shortstack{ - 7-13 ans \\ - Diagnostiqué \gls{tdah} \\ - Enregistré avec \\ l'appareil
                                               \\ Mensia Koala\textregistered : \\ 8 électrodes \\ du système 10-20 } 
																							 & \shortstack{ - 6-44 ans \\ - Avec et sans \\ diagnostic \\ - Enregistré avec \\ le système
                                               \gls{eeg} \\ Geodesic Hydrocel : \\ 128 électrodes } 
																							 & \multicolumn{2}{ c }{ \shortstack{ - 5-21 ans \\ - Avec et sans \\ diagnostic \\ - Enregistré avec \\ le système
                                               \gls{eeg} \\ Geodesic Hydrocel : \\ 128 électrodes } }
																							\\
\midrule
Nombre de sujets & \shortstack{ 122 (données disponibles \\ au 09/2017 pour les \\ analyses de contrôle \\ de la qualité avant \\ la fin de l'étude \\ en 12/2017) } 
                 & 126 
								 & \multicolumn{2}{ c }{ 881 }
								\\
\midrule
\shortstack{ Critères d'inclusion \\ additionnels} & \shortstack{ 1. Age/diagnostic \\ précisés (122) \\ 2. \textbf{Diagnostic} \\ \textbf{ \gls{tdah} } (122) \\ 3. \gls{eeg} d'une \\ min \gls{eo} 
                 au repos \\ disponible et \\ possible à \\ analyser (122) } 
                 & \shortstack{ 1. Age/diagnostic \\ précisés (126) \\ 2. \textbf{Diagnostic :} \\ \textbf{ \gls{tdah} } (12) \\ 3. \gls{eeg} d'une \\ min \gls{eo} 
                 au repos \\ disponible et \\ possible à \\ analyser (10) } 
								 & \shortstack{ 1. Age/diagnostic \\ précisés (447) \\ 2. \textbf{Diagnostic :} \\ \textbf{ \gls{tdah} } (237) \\ 3. \gls{eeg} d'une \\ min \gls{eo} 
                 au repos \\ disponible et \\ possible à \\ analyser (231) } 
								 & \shortstack{ 1. Age/diagnostic \\ précisés (447) \\ 2. \textbf{Diagnostic :} \\ \textbf{ Aucun }  (76) \\ 3. \gls{eeg} d'une \\ min \gls{eo}  
                 au repos \\disponible et \\ possible à \\ analyser (74) } 
								\\
\midrule
\shortstack{ Nombre de \\ sujets inclus} & 122 & 10 & 231 & 74 \\
\bottomrule
\end{tabular}
  \label{Table:tbr_datasets_description}
\end{table}

\subsection{Données NEWROFEED}

Une partie des données utilisées dans cette analyse provient de l'étude NEWROFEED (NCT02778360, Mensia Technologies, France, ClinicalTrials.gov, \citet{Bioulac2019})
qui avait pour but d'évaluer l'efficacité du \gls{nfb} à la maison versus celle du méthylphenidate sur une population d'enfants \gls{tdah}.
Au moment où le travail décrit dans ce chapitre a été mené, NEWROFEED était en cours donc 
seulement une partie des données était disponible. Ainsi, 122 enregistrements \gls{eeg} d'enfants diagnostiqués \gls{tdah} d'après les critères du DSM-IV \citep{DSM-4} 
ont été analysés. L'\gls{eeg} a été enregistré avec l'appareil Mensia Koala équipé de 8 électrodes \gls{agcl} individuellement blindées, positionnées sur le scalp suivant
le système international 10-20 : Fpz, F3, Fz, F4, C3, Cz, C4, Pz. La fréquence d'échantillonnage était de 512Hz. Les impédances devaient être
inférieures à $40$k$\Omega$ et le niveau de contamination électromagnétique devait rester inférieur à 1/3 de l'énergie totale du signal. 

Pour participer à l'étude NEWROFEED, les sujets devaient remplir les critères suivants :
\begin{itemize}
\item être des enfants ou adolescents (fille ou garçon) entre 7 et 13 ans,
\item avoir un diagnostic \gls{tdah} positif avec Kiddie-SADS \citep{Kaufman1997},
\item avoir un score sur l'ADHD RS IV supérieur à 6 pour l'inattention, avec ou sans hyperactivité \citep{Pappas2006}.
\end{itemize}

De plus, les enfants correspondant à un de ces critères ont été exclus :
\begin{itemize}
\item être \gls{tdah} avec le sous-type hyperactif/impulsif mais sans la composante inattention,
\item avoir un trouble psychiatrique sevère et/ou incontrôlable autre que le \gls{tdah} diagnostiqué avec Kiddie-SADS tel que par 
exemple l'autisme ou la schizophrénie,
\item avoir un trouble comorbide nécessitant des médicaments psychoactifs autres que ceux prescrits pour le \gls{tdah},
\item avoir un QI < 80 d'après les trois sous-tests du WASI ou du WISC \citep{Wechsler1999}.
\end{itemize}

Seule la première évaluation de l'\gls{eeg} enregistrée pour chaque patient avant le début du traitement par \gls{nfb} est utilisée pour cette analyse. 
L'étude NEWROFEED a été menée dans 12 centres cliniques dans 5 pays européens (France, Espagne, Allemagne, Belgique et Suisse).

\subsection{Données CMI-MIPDB}
Au moment de l'analyse, l'intégralité de la base \gls{cmi-mipdb} compte 126 participants à la fois avec et sans diagnostic clinique \citep{Langer2017, Langer2017b}.
Les participants ont été recrutés au Child Mind Medical Practice et dans la région de la ville de New-York. Chaque sujet a été questionné pendant 10 minutes
au téléphone ou en personne par un chercheur assistant expérimenté pour évaluer son éligibilité grâce à :
\begin{itemize}
\item l'historique de ses troubles psychiatriques, incluant les traitements en cours et précédents,
\item l'historique de ses troubles neurologiques et/ou épilepsie.
\end{itemize}

L'enregistrement de l'\gls{eeg} est prévu si aucune contre indication n'est trouvée. 

De tous les patients présents dans la base \gls{cmi-mipdb}, seulement ceux satisfaisant les critères suivants ont été inclus dans l'analyse présentée dans ce chapitre :
\begin{enumerate}
\item être diagnostiqué\gls{tdah},
\item posséder un \gls{eeg} au repos disponible au format ".raw",
\item avoir son âge précisé.
\end{enumerate}

L'\gls{eeg} a été enregistré avec un système \gls{eeg} Geodesic Hydrocel à une fréquence d'échantillonnage de 500Hz et un filtre passe-bande entre 0.1 et 100Hz. 
L'électrode de référence est Cz, localisée au vertex de la tête. Le tour de tête de chaque participant est mesuré pour que le bonnet utilisé lors de l'enregistrement 
soit à la bonne taille. L'impédance des électrodes est gardée inférieure à $40$k$\Omega$ : elle est vérifiée toutes les 30 minutes ainsi qu'avant chaque enregitrement.

\subsection{Données CMI-HBN}
La base de données \gls{cmi-hbn} est composée de 881 sujets, avec ou sans diagnostic \citep{Alexander2017, Alexander2017b}. Les fammilles recoivent 150\$ pour leur participation 
et les sujets se voient de plus offrir les rapports de consultation et les avis sur les sessions d'\gls{eeg}.

Seuls les sujets remplissant les critères suivants ont été inclus dans notre analyse :
\begin{enumerate}
\item être diagnostiqué \gls{tdah} selon le KSADS-COMP \citep{Kaufman1997},
\item posséder un \gls{eeg} au repos disponible au format matlab ".mat",
\item avoir son âge précisé.
\end{enumerate}

Des 881 sujets disponibles, 231 (âgés entre 5 et 21 ans) satisfont ces critères et sont inclus dans notre analyse.

La base de données \gls{cmi-hbn} contient également des sujets sains qui vont être utilisés en tant qu'a priori (\textit{priors} en anglais) pour le
modèle bayesien décrit en \ref{bgmm}. Pour être inclus dans cette analyse en tant que \textit{priors}, les sujets ne doivent avoir aucun diagnostic 
et doivent remplir les critères 2. et 3. cités précédemment. Au final, 74 sujets entre 5 et 21 ans sont sélectionnés. 

\subsection{Pré-traitement et homogénéisation des bases de données}
Les \gls{eeg} des différentes bases de données sont pré-traités de façon à être comparables, notamment au niveau du placement des électrodes.
En ce qui concerne le traitement des artefacts et l'extraction du \gls{tbr}, les étapes sont les mêmes quelle que soit la base de données. 
Les pré-traitements ainsi que les analyses des signaux \gls{eeg} qui suivent sont effectués à l'aide du logiciel NeuroRT (v3, Mensia Technologies, 
Paris, France).

\subsubsection{Base de données NEWROFEED}
Le seul pré-traitement que nécessite les signaux de la base de données NEWROFEED est un filtrage temporel : 
\begin{itemize}
\item un filtre Butterworth passe-haut d'ordre 1 à 0.5Hz afin d'enlever la composante continue (\textit{DC component} en anglais),
\item un filtre Butteroworth coupe-bande d'ordre 3 de 47 à 53Hz afin d'enlever l'artefact causé par les lignes électriques.
\end{itemize}

\subsubsection{Bases de données CMI}

Le pré-traitement des bases de données \gls{cmi-mipdb} et \gls{cmi-hbn} demande plus d'étapes : filtrage temporel, suppression et 
interpolation des canaux bruités et/ou déconnectés, et une interpolation spatiale afin de passer d'un espace à 128 électrodes \gls{egi},
à l'espace de 8 électrodes placées selon le système 10-20 utilisé dans la base de données NEWROFEED.

Tout d'abord, les \gls{eeg} obtenus au repos sont débruités et séparés en deux fichiers : l'un pour les enregistrements les yeux fermés, 
l'autre pour les enregistrements \gls{eo} ; seul ce dernier va être analysé. Ensuite, les mêmes filtres temporels vont être appliqués
que pour les données NEWROFEED, à l'exception du filtre coupe bande dont les bornes sont modifiées pour intercepter les artefacts 
causés par les lignes électriques américaines (57-63Hz).

Dans le cas d'enregistrements d'\gls{eeg} avec une haute résolution spatiale comme c'est le cas avec les données CMI, 
il est courant qu'au moins un canal se déconnecte ponctuellement causant aussi bien des artefacts de large amplitude que des signaux plats. 
Une strategie est ici mise en place pour détecter de façon fiable puis interpoler ces électrodes déconnectées. La variance de chaque canal 
\gls{eeg} est calculée sur une fenêtre glissante de 10 secondes. Cette variance est ensuite convertie en deux z-scores grâce à :
\begin{enumerate}
\item la distribution instantanée des variances pour les 127 autres canaux (z-score spatial),
\item la distribution cumulative des variances pour le canal d'intérêt (z-score temporel).
\end{enumerate}
Si le z-score temporel n'est pas compris entre -5 et 5, le signal est détecté comme étant un artefact et est interpolé à partir des canaux voisins.
Si le z-score spatial n'est pas compris entre -2 et 2, soit le signal a une trop grande variance (autrement dit il est bruité), soit une trop faible
variance (autrement dit l'électrode doit être déconnectée et n'energistre rien) : dans les deux cas il est interpolé.

\section{Theta-Beta ratio : un marqueur de l'attention}

\subsection{Définition du Theta-Beta ratio}
Articles scientifiques sur le TBR chez les enfants TDAH, marqueur de l'attention, parler de NebaHealth

\subsection{Extraction du Theta-Beta ratio}

\section{Méthodes de modélisation de la distribution des Theta-Beta ratios}

\subsection{Partitionnement basé sur les distributions : Bayesian Gaussian Mixture Model} \label{bgmm}

\subsection{Partitionnement basé sur les distances : méthode de Ward}

\subsection{Partitionnement basé sur les densités : DBSCAN}

\subsection{Identification du seuil optimal pour la personnalisation du protocole de Neurofeedback}

\section{Partitionnement de la distribution des Theta-Beta ratios}

\subsection{Partitionnements obtenus}

\subsection{Seuils identifiés}

\section{Discussion}

\subsection{Groupes et seuils identifiés}

\subsection{Analyse des facteurs de confusion}

\subsubsection{Correction des artefacts oculaires}

\subsubsection{Influence de l'âge}