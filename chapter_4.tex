\chapter{Analyse de la distribution d'un marqueur de l'attention au sein d'une population d'enfants TDAH}

\section*{Introduction}
Diminuer le \gls{tbr} est un protocole d'entrainement de \gls{nfb} couramment utilisé pour traiter le \gls{tdah} chez les enfants 
\citep{Arnold2014, Deilami2016, Gevensleben2009, VanDongen2013}. Cependant, ce protocole ne serait peut-être pas adapté à tous les enfants \gls{tdah} si 
on se base sur le phénotype de leur \gls{eeg}. En effet, il a été avancé qu'il existerait un groupe d'enfants \gls{tdah} présentant un \gls{tbr} élevé 
\citep{Zhang2017, Clarke2011}, ainsi ces enfants bénéficiraient peut-être davantage d'un protocole diminuant leur \gls{tbr} que les autres. 

Quelques études ont proposé de personnaliser le protocole d'entrainement par \gls{nfb} \citep{Bazanova2018, Escolano2014}, mais trop peu pour en déterminer
l'impact sur l'efficacité du \gls{nfb} par la \gls{saob}. L'analyse présentée dans ce chapitre n'a pas pour but d'évaluer directement l'efficacité de la 
personnalisation des protocoles de \gls{nfb} mais sa pertinence. Pour ce faire, la distribution des valeurs de \gls{tbr} chez les enfants \gls{tdah} est 
étudiée à l'aide de différentes méthodes de partitionnement. 

% à revoir

\section{Population étudiée}

Les données utilisées dans cette analyse proviennent de trois bases de données différentes :
\begin{itemize}
\item NEWROFEED (NCT02778360, Mensia Technologies, France, ClinicalTrials.gov, \citep{Bioulac2019}),
\item Child Mind Institute Multimodel Resource for Studying Information Processing in the Developing Brain (CMI-MIPDB) \citep{Langer2017, Langer2017b},
\item Child Mind Institute Healthy Brain Network (CMI-HBN) \citep{Alexander2017, Alexander2017b}.
\end{itemize}
Pour chacune de ces bases de données, un consentement éclairé écrit a été obtenu de tous les participants ou de leurs responsables légaux. Tous les enregistrements
ont été effectués dans un environnement contrôlé avec les yeux ouverts (\gls{eo} en anglais) et au repos pendant une minute sous la supervision d'un clinicien ou d'un chercheur. 

La description de l'ensemble des données est disponible dans la Table~\ref{Table:tbr_datasets_description}.

\begin{table}[h!]
  \centering
  \caption{Informations sur les données utilisées. Le critère d'inclusion pour chaque base de données est précisé, ainsi que le nombre de sujets satisfaisant
	chaque critère entre parenthèses. Le nombre total de sujets inclus par base de données est donné à la dernère ligne.}
  \fontsize{9}{11}\selectfont
\begin{tabular}{ ccccc }
\toprule
Base de données & NEWROFEED & CMI-MIPDB & \multicolumn{2}{ c }{CMI-HBN} \\
\midrule
\shortstack{ Description \\ de la population } & \shortstack{ - 7-13 ans \\ - Diagnostiqué \gls{tdah} \\ - Enregistré avec \\ l'appareil
                                               \\ Mensia Koala\textregistered : \\ 8 électrodes \\ du système 10-20 } 
																							 & \shortstack{ - 6-44 ans \\ - Avec et sans \\ diagnostic \\ - Enregistré avec \\ le système
                                               \gls{eeg} \\ Geodesic Hydrocel : \\ 128 électrodes } 
																							 & \multicolumn{2}{ c }{ \shortstack{ - 5-21 ans \\ - Avec et sans \\ diagnostic \\ - Enregistré avec \\ le système
                                               \gls{eeg} \\ Geodesic Hydrocel : \\ 128 électrodes } }
																							\\
\midrule
Nombre de sujets & \shortstack{ 122 (données disponibles \\ au 09/2017 pour les \\ analyses de contrôle \\ de la qualité avant \\ la fin de l'étude \\ en 12/2017) } 
                 & 126 
								 & \multicolumn{2}{ c }{ 881 }
								\\
\midrule
\shortstack{ Critères d'inclusion \\ additionnels} & \shortstack{ 1. Age/diagnostic \\ précisés (122) \\ 2. \textbf{Diagnostic} \\ \textbf{ \gls{tdah} } (122) \\ 3. \gls{eeg} d'une \\ min \gls{eo} 
                 au repos \\ disponible et \\ possible à \\ analyser (122) } 
                 & \shortstack{ 1. Age/diagnostic \\ précisés (126) \\ 2. \textbf{Diagnostic :} \\ \textbf{ \gls{tdah} } (12) \\ 3. \gls{eeg} d'une \\ min \gls{eo} 
                 au repos \\ disponible et \\ possible à \\ analyser (10) } 
								 & \shortstack{ 1. Age/diagnostic \\ précisés (447) \\ 2. \textbf{Diagnostic :} \\ \textbf{ \gls{tdah} } (237) \\ 3. \gls{eeg} d'une \\ min \gls{eo} 
                 au repos \\ disponible et \\ possible à \\ analyser (231) } 
								 & \shortstack{ 1. Age/diagnostic \\ précisés (447) \\ 2. \textbf{Diagnostic :} \\ \textbf{ Aucun }  (76) \\ 3. \gls{eeg} d'une \\ min \gls{eo}  
                 au repos \\disponible et \\ possible à \\ analyser (74) } 
								\\
\midrule
\shortstack{ Nombre de \\ sujets inclus} & 122 & 10 & 231 & 74 \\
\bottomrule
\end{tabular}
  \label{Table:tbr_datasets_description}
\end{table}

\subsection{Données NEWROFEED}

Au moment où cette étude a été menée, l'étude NEWROFEED était en cours donc seulement une partie des données était disponibles. Ainsi, 122 enregistrements 
\gls{eeg} d'enfants diagnostiqués \gls{tdah} en se basant sur le critère du DSM-IV \citep{DSM-4} ont été analysés. L'\gls{eeg} a 
été enregistré avec l'appareil Mensia Koala équipé de 8 électrodes \gls{agcl} individuellement blindées positionnées sur le scalp suivant
le système international 10-20 : Fpz, F3, Fz, F4, C3, Cz, C4, Pz. La fréquence d'échantillonnage était de 512Hz. Les impédances devaient être
inférieures à $40$k$\Omega$ et le niveau de contamination électromagnétique devait rester inférieur à 1/3 de l'énergie totale du signal. 

Pour entrer dans l'étude, les sujets devaient remplir les critères suivants :
\begin{itemize}
\item enfants ou adolescents (fille ou garçon) entre 7 et 13 ans,
\item diagnostic \gls{tdah} positif avec Kiddie-SADS \citep{Kaufman1997},
\item ADHD RS IV supérieure à 6 pour l'inattention, avec ou sans hyperactivité \citep{Pappas2006}.
\end{itemize}

De plus, les enfants correspondant à un de ces critères ont été exclus :
\begin{itemize}
\item être \gls{tdah} avec le sous-type hyperactif/impulsif mais pas la composante inattention,
\item avoir trouble psychiatrique sevère et/ou incontrôlable autre que le \gls{tdah} diagnostiqués avec Kiddie-SADS tel que par 
exemple l'autisme, la schizophrénie,
\item avoir un trouble comorbide nécessitant des médicaments psychoactifs autres que ceux prescrits pour le \gls{tdah},
\item avoir un QI < 80 d'après les trois sous-tests du WASI ou du WISC \gls{Wechsler1999}.
\end{itemize}

Seule la première évaluation de l'\gls{eeg} enregistré pour chaque patient est utilisée pour cette analyse. La première évaluation est 
enregistrée à l'hôpital sous la supervision d'un clinicien, au repos (c'est à dire que le sujet n'effectue aucune tâche) durant une minute 
les yeux ouverts. L'étude NEWROFEED a été menée dans 12 centres cliniques dans 5 pays européens (France, Espagne, Allemagne, Belgique et Suisse).



\subsection{Données CMI-HBN}

\subsection{Données CMI-MIPDB}

\subsection{Pré-traitement et homogénéisation des bases de données}

\section{Theta-Beta ratio : un marqueur de l'attention}

\subsection{Définition du Theta-Beta ratio}
Articles scientifiques sur le TBR chez les enfants TDAH

\subsection{Extraction du Theta-Beta ratio}

\section{Modélisation de la distribution des Theta-Beta ratios}

\subsection{Partitionnement basé sur les distributions : Bayesian Gaussian Mixture Model}

\subsection{Partitionnement basé sur les distances : méthode de Ward}

\subsection{Partitionnement basé sur les densités : DBSCAN}

\subsection{Identification du seuil optimal pour la personnalisation du protocole de Neurofeedback}

\section{Analyse des facteurs de confusion}

\subsection{Correction des artefacts oculaires}

\subsection{Influence de l'âge}