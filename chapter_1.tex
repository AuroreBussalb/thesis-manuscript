\chapter{Introduction} \label{chapitre-1}


\section{Définition du Neurofeedback}
% voir lambez B et al. 2019

\subsection{Principe du Neurofeedback}

Le \gls{nfb} est une technique de thérapie comportementale ayant pour but d'apprendre ou d'améliorer l'auto-régulation de l'activité cérébrale.

% utiliser Marzbani
Conditionnement opérant (à opposer au conditionement classique), définition phase de transfert, type de seuillage, individualisation des protocoles (iapf), 
définition rapide iapf
montage, matériel utilisé, marquage CE ou non, gestion des artefacts, définition neuromarqueur, définition impédance
définition des bandes de fréquence


\subsection{Historique}

Les prémices du \gls{nfb} remontent au début des années 1930, peu après l'enregistrement du premier \gls{eeg} humain par Hans Berger en 1929.
En effet, \citet{Durup1935} et \citet{Loomis1936} ont observé que les ondes alpha, osciallant entre 8 et 12Hz, 
pouvaient être contrôlées grâce au conditionnment classique. 

Plus tard, ce qui peut être considéré comme le premier entrainement par \gls{nfb} a été mené par le Dr. Kamiya qui a demandé aux sujets de son étude de contrôler le rythme 
alpha et qui a obtenu des résultats prometteurs \citep{Kamiya1969}. 

Une preuve solide de la modulation de l'\gls{eeg} a été rapportée dans les années 1960 par le Dr. Sterman : l'expérience 
originelle consistait à entrainer le cerveau des chats en leur apprenant quoi faire pour obtenir de 
la nourriture. Durant cette expérience, le Dr. Sterman a extrait un rythme particulier au niveau du cortex sensorimoteur : le \gls{smr}, qui correspond aux fréquences 
entre 12 et 15Hz. Ensuite, le but du Dr. Sterman a été d'entrainer les chats à produire ce rythme en leur donnant de la nourriture lorsqu'ils réussissaient
à l'émettre pendant une demi seconde, ce que les chats ont vite appris à faire. 

Cette modulation a ensuite été employée à des fins cliniques : l'entrainement du \gls{smr} chez les chats améliore leur sommeil \citep{Sterman1970},
résultat reproduit chez l'humain \citep{Hoedlmoser2008}. Cependant, le \gls{nfb} ne correspondant pas à ce qui était connu à l'époque quant au fonctionnement
du cerveau humain, son utilisation s'est faite en marge de la communauté scientifique \citep{Masterpasqua2003}. Il faut attendre les années 2000 pour que le 
\gls{nfb} soit réhabilité conduisant à une explosion du nombre d'études scientifiques sur ce sujet, comme illustré à la 
Figure~\ref{Figure:introduction_number_of_nfb_publications}, visant à mieux comprendre ses mécanismes et ses effets. 

\begin{figure}[h!]
  \centering
	\includegraphics[width=0.7\linewidth]{figures/chapter-1/introduction-number-of-nfb-publications} 
  \caption{Evolution du nombre de publications sur le Neurofeedback par année. La base de données PubMed a été questionnée avec les 
	termes de recherche "Neurofeedback OR EEG Biofeedback".}
  \label{Figure:introduction_number_of_nfb_publications}
\end{figure}

\section{Les champs d'application du Neurofeedback}

\subsection{De nombreuses applications}
Epilepsie, diminution de l’anxiété, douleurs chroniques (citer papier de louis et quentin), etc

\subsection{Neurofeedback et \gls{tdah}}
Définition TDAH chez l’enfant (parler du dsm-4 et 5) et parler de l’essor de la problématique du TDAH chez l’adulte, parler des études et des méta-analyses
sham-NFB, parler des échelles cliniques (It is generally accepted that parent and teacher rating scales are reliable and 
valid components of ADHD assessments (McGough  Barkley, 2004)), définir pblind et mprox
Parler des traitements possibles (bien définir les cognitives therapy)

\section{Objectifs de la thèse}
Le \gls{nfb} a fait l'objet de nombreuses études pour déterminer son efficacité dans le cadre du \gls{tdah} chez l'enfant comme souligné précédemment.
Malheureusement, aucun consensus n'a encore été clairement atteint, ainsi le travail effectué au cours de cette thèse a pour but de déterminer les facteurs 
de réussite de l'entraînement par \gls{nfb} pour les enfants \gls{tdah} en se basant sur des données cliniques mais aussi physiologiques. 

Les trois sous objectifs de ce travail, chacun développé dans un chapitre, sont les suivants :
\renewcommand{\labelitemi}{$\bullet$}
\renewcommand{\labelitemii}{$\cdot$}
\begin{itemize}
\item étudier l'efficacité du \gls{nfb} chez les enfants \gls{tdah} à l'aide d'une méthode couramment utilisée : la méta-analyse,
\item identifier les paramètres méthodologiques et cliniques influençant la performance de ce traitement,
\item analyser la distribution d'un marqueur de l'attention au sein d'une population d'enfants \gls{tdah} pour mieux cibler
l'entrainement par \gls{nfb}. 
\end{itemize}

\section{Contribution et résumé des chapitres}

Ce manuscrit est divisé en 5 parties : les chapitres \ref{chapitre-2}, \ref{saob} et \ref{chapitre-4} ont chacun pour but de remplir un des objectifs précédemment énoncés.

Le chapitre \ref{chapitre-2} s'intéresse à une méthode largement utilisée pour évaluer la performance du \gls{nfb} pour les enfants \gls{tdah} 
\citep{Sonuga-Barke2013, Micoulaud2014, Cortese2016} : la méta-analyse. Les résultats de ce type d'analyse ont un impact important sur la
communauté scientifique : \citet{Micoulaud2016} a notamment réagi à la méta-analyse de \citet{Cortese2016} en discutant certains points de cette 
analyse. 

Ainsi, dans ce chapitre la méta-analyse de \citet{Cortese2016} est répliquée en modifiant les points soulignés par \citet{Micoulaud2016}
afin de jauger leur impact sur les conclusions émises dans la méta-analyse. Ensuite, étant donné que de nouvelles études satisfaisant les critères d'inclusion
établis par \citet{Cortese2016} sont disponibles, cette méta-analyse est mise à jour : en plus des 13 études originellement incluses, 3
sont ajoutées, ce qui apporte une plus grande puissance statistique aux résultats. La réplication et la mise à jour sont effectuées, non pas avec les logiciels 
habituellement utilisés tels que Revman \citep{Revman}, mais à l'aide d'un package Python développé pour cette occasion et disponible en ligne afin de favoriser 
la réplication et/ou la mise à jour de ce travail. 

La réplication de la méta-analyse conduisant aux mêmes résultats que \citet{Cortese2016}, les choix discutés par \citet{Micoulaud2016} n'ont pas un impact assez
important pour changer ses conclusions : le \gls{nfb} est jugé efficace par les parents alors que les enseignants, considérés comme \gls{pblind}, ne notent aucune
amélioration significative. Par ailleurs, la mise à jour confirme les résultats qui semblent commencer à se stabiliser.

Ce chapitre a été l'occasion de mener une revue de littérature des études cliniques sur le \gls{nfb} appliqué aux enfants \gls{tdah} 
qui a permis de mettre en évidence la forte hétérogénéité d'un point de vue clinique et méthodologique de ces études. 

Alors que les résultats des méta-analyses peuvent souffrir de ces différences qui pourraient, par ailleurs, expliquer l'absence de consensus quant 
à l'efficacité du \gls{nfb}, une analyse en tirant avantage est implémentée : la \gls{saob} décrite dans le chapitre \ref{saob}. Les facteurs méthodologiques et/ou 
cliniques fortement variables entre les études tels que, par exemple, la durée du traitement, le nombre de sessions et le type de protocole de \gls{nfb} 
suivi, sont extraits de 33 études d'efficacité sur le \gls{nfb} appliqué aux enfants \gls{tdah} dans le but de déterminer lesquels ont un impact sur 
l'efficacité du \gls{nfb}. Pour ce faire, trois méthodes multivariées sont utilisées : la \gls{wls}, le \gls{lasso} et le \gls{dt}. 

La \gls{saob} identifie trois facteurs qui semblent avoir un impact sur l'efficacité du \gls{nfb} : tout d'abord, utiliser un matériel d'acquisition de bonne 
qualité conduirait à de meilleurs résultats, ensuite un traitement intensif semblerait préférable, et enfin les évaluations des enseignants seraient plus
sévères quant à l'amélioration du traitement. 

La personnalisation des protocoles de \gls{nfb} est un facteur dont il aurait été intéressant d'étudier l'impact sur l'efficacité du \gls{nfb}.  
Cependant, faute d'un nombre suffisant d'études ayant recours à la personnalisation, ce facteur n'a pas pu être étudié dans la \gls{saob}. C'est pourquoi
la pertinence d'une personnalisation est étudiée dans le chapitre \ref{chapitre-4} grâce à l'analyse de la distribution d'un marqueur de l'attention : le \gls{tbr} pour
lequel de précédentes études ont montré qu'il serait variable au sein de la population \gls{tdah} \citep{Zhang2017, Arns2013, Clarke2001}.

Le \gls{tbr} est extrait de 363 \gls{eeg} d'enfants \gls{tdah} qui vont être partitionnés grâce à trois méthodes : le \gls{bgmm}, le partitionnement
hiérarchique basé sur la distance de Ward et le \gls{dbscan}. Si la distribution est trouvée bimodale par ces trois méthodes, le seuil séparant
les deux modes sont étudiés afin de déterminer le seuil \gls{tbr} optimal sur lequel l'attribution du protocole de \gls{nfb} pourrait se baser.

Les trois méthodes s'accordent sur le fait que la distribution des \gls{tbr} est bimodale, ce qui indique qu'il existe en effet deux groupes d'enfants
\gls{tdah} : l'un présentant des \gls{tbr} plutôt faibles et l'autre avec des \gls{tbr} élevés. Ainsi, personnaliser le protocole de \gls{nfb} en 
fonction de la valeur de \gls{tbr} semblerait pertinent, toutefois il reste à déterminer le seuil optimal permettant d'attribuer le protocole le plus
adapté. Pour ce faire, les seuils de séparation calculés pour chacune des méthodes sont comparés notamment grâce à une courbe \gls{roc} calculée sur
les résultats du \gls{bgmm} : un seuil de 4.1 est celui apportant de plus d'équilibre entre un faible taux de faux positifs et un taux de vrais positifs 
élevé.  



%Lister les objectifs de la thèse :
%- Réplication et mise à jour d’une méta analyse en codant les étapes en Python
%- Identification des facteurs ayant une influence sur le NFB grâce à des méthodes multivariées
%- Analyse de la distribution d’un marqueur de l’attention pour augmenter l’efficacité du NFB
%
%dire que ces chapitres se basent sur les résultats des articles publiés mais aussi mis à jour
%
%Méthodes utilisées : méta-analyse, méthode multivariées et méthodes de partitionnement
%- Résultats
%- Discussion

\section{Liste des publications}

Le travail décrit dans ce manuscrit a donné lieu aux publications avec comité de lecture suivantes :

\begin{description}
\item \citet{Bussalb2019tbr} : A. Bussalb, S. Collin, Q. Barthélemy, D. Ojeda, E. Acquaviva, S. Bioulac, H. Blasco-Fontecilla,
D. Brandeis, R. Delorme, D. P. Ouakil, T. Ros, and L. Mayaud. Is there a cluster of high
theta-beta ratio patients in attention deficit hyperactivity disorder ? \textit{Clinical Neurophysiology}, 2019a.
\item \citet{Bussalb2019clinical} : A. Bussalb, M. Congedo, Q. Barthélemy, D. Ojeda, E. Acquaviva, R. Delorme,
and L. Mayaud. Clinical and experimental factors influencing the efficacy of
neurofeedback in ADHD: a meta-analysis. \textit{Frontiers in psychiatry}, 10 :35, 2019b.
\end{description}

Une partie des travaux de \citet{Bussalb2019clinical} ont fait l'objet d'une communication orale :

\noindent A. Bussalb, M. Congedo, R. Delorme, E. Acquaviva, Q. Barthelemy, D. Ojeda, J.A. Micoulaud-Franchi, L. Mayaud. Neurofeedback 
appliqué aux enfants TDAH : quels facteurs influencent son efficacité ? 6ème Congrès de la SOFTAL, mai 2018. 


