\chapter{Introduction}


\section{Définition du Neurofeedback}
% voir lambez B et al. 2019
\subsection{Historique}
Découverte de l’EEG, premiers pas de la neuromodulation avec Durup 
et Fressard, les études de Sterman et Lubar, la très forte augmentation des études sur le Neurofeedback depuis 
les années 2000

\subsection{Principe du Neurofeedback}
Conditionnement opérant, définition phase de transfert, type de seuillage, individualisation des protocoles (iapf), 
définition rapide iapf
montage, matériel utilisé, marquage CE ou non, gestion des artefacts, définition neuromarqueur, définition impédance
définition des bandes de fréquence

\section{Les champs d'application du Neurofeedback}

\subsection{De nombreuses applications}
Epilepsie, diminution de l’anxiété, douleurs chroniques, etc

\subsection{Neurofeedback et \gls{tdah}}
Définition TDAH chez l’enfant (parler du dsm-4 et 5) et parler de l’essor de la problématique du TDAH chez l’adulte, parler des études et des méta-analyses
sham-NFB, parler des échelles cliniques (It is generally accepted that parent and teacher rating scales are reliable and 
valid components of ADHD assessments (McGough  Barkley, 2004)), définir pblind et mprox
Parler des traitements possibles (bien définir les cognitives therapy)

\section{Objectifs de la thèse}
Le \gls{nfb} a fait l'objet de nombreuses études pour déterminer son efficacité dans le cadre du \gls{tdah} chez l'enfant comme souligné précédemment.
Malheureusement, aucun consensus n'a encore été clairement atteint, ainsi le travail effectué au cours de cette thèse a pour but de déterminer les facteurs 
de réussite de l'entraînement par \gls{nfb} pour les enfants \gls{tdah} en se basant sur des données cliniques mais aussi physiologiques. 

Les trois sous objectifs de ce travail, chacun développé dans un chapitre, sont les suivants :
\renewcommand{\labelitemi}{$\bullet$}
\renewcommand{\labelitemii}{$\cdot$}
\begin{itemize}
\item étudier l'efficacité du \gls{nfb} chez les enfants \gls{tdah} à l'aide d'une méthode couramment utilisée : la méta-analyse,
\item identifier les paramètres méthodologiques et cliniques influençant la performance de ce traitement,
\item analyser la distribution d'un marqueur de l'attention au sein d'une population d'enfants \gls{tdah} pour mieux cibler
l'entrainement par \gls{nfb}. 
\end{itemize}

La réponse à ces questions 

\section{Contribution et résumé des chapitres}

Ce manuscrit est divisé en trois chapitres, dont chacun a pour but de remplir un des objectifs précédemment énoncés.

Le premier chapitre s'intéresse à une méthode largement utilisée pour évaluer la performance du \gls{nfb} pour les enfants \gls{tdah} 
\citep{Sonuga-Barke2013, Micoulaud2014, Cortese2016} : la méta-analyse. Les résultats de ce type d'analyse ont un impact important sur la
communauté scientifique : \citet{Micoulaud2016} a notamment réagi à la méta-analyse de \citet{Cortese2016} en discutant certains points de cette 
analyse. 

Ainsi, dans ce chapitre la méta-analyse de \citet{Cortese2016} est répliquée en modifiant les points soulignés par \citet{Micoulaud2016}
afin de jauger leur impact sur les conclusions de la méta-analyse. Ensuite, étant donné que de nouvelles études satisfaisant les critères d'inclusion
établis par \citet{Cortese2016} sont disponibles depuis la fin de leurs recherches, cette méta-analyse est mise à jour ce qui apporte une plus grande puissance
statistique aux résultats. La réplication et la mise à jour sont effectuées, non pas avec les logiciels habituellement utilisés tels que Revman \citep{Revman},
mais à l'aide d'un package Python développé pour cette occasion. 

La réplication de la méta-analyse conduisant aux mêmes résultats que \citet{Cortese2016}, les choix discutés par \citet{Micoulaud2016} n'ont pas un impact assez
important pour changer ses conclusions. Par ailleurs, la mise à jour confirme les résultats qui semblent se stabiliser.

Ce premier chapitre a été l'occasion de mener une revue de littérature des études cliniques sur le \gls{nfb} appliqué aux enfants \gls{tdah}. 
Cette revue de littérature a permis de mettre en évidence la forte hétérogénéité d'un point de vue clinique et méthodologique de ces études.
Alors que les résultats des méta-analyses peuvent souffrir de ces différences, une analyse en tirant avantage est implémentée : la \gls{saob}.
Cette analyse comprend trois méthodes multivariées qui tentent d'expliquer l'efficacité du \gls{nfb} à l'aide de paramètres méthodologiques et 
cliniques. 



Lister les objectifs de la thèse :
- Réplication et mise à jour d’une méta analyse en codant les étapes en Python
- Identification des facteurs ayant une influence sur le NFB grâce à des méthodes multivariées
- Analyse de la distribution d’un marqueur de l’attention pour augmenter l’efficacité du NFB

dire que ces chapitres se basent sur les résultats des articles publiés mais aussi mis à jour

Méthodes utilisées : méta-analyse, méthode multivariées et méthodes de partitionnement
- Résultats
- Discussion

\section{Liste des publications}

Le travail décrit dans ce manuscrit a donné lieu aux publications avec comité de lecture suivantes :

\begin{description}
\item \citet{Bussalb2019tbr} : A. Bussalb, S. Collin, Q. Barthélemy, D. Ojeda, E. Acquaviva, S. Bioulac, H. Blasco-Fontecilla,
D. Brandeis, R. Delorme, D. P. Ouakil, T. Ros, and L. Mayaud. Is there a cluster of high
theta-beta ratio patients in attention deficit hyperactivity disorder ? \textit{Clinical Neurophysiology}, 2019a.
\item \citet{Bussalb2019clinical} : A. Bussalb, M. Congedo, Q. Barthélemy, D. Ojeda, E. Acquaviva, R. Delorme,
and L. Mayaud. Clinical and experimental factors influencing the efficacy of
neurofeedback in ADHD: a meta-analysis. \textit{Frontiers in psychiatry}, 10 :35, 2019b.
\end{description}

Une partie des travaux de \citet{Bussalb2019clinical} ont fait l'objet d'une communication orale :

\noindent A. Bussalb, M. Congedo, R. Delorme, E. Acquaviva, Q. Barthelemy, D. Ojeda, J.A. Micoulaud-Franchi, L. Mayaud. Neurofeedback 
appliqué aux enfants TDAH : quels facteurs influencent son efficacité ? 6ème Congrès de la SOFTAL, mai 2018. 
