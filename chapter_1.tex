\chapter{Introduction}


\section{Définition du Neurofeedback}

\subsection{Historique}
Découverte de l’EEG, premiers pas de la neuromodulation avec Durup 
et Fressard, les études de Sterman et Lubar, la très forte augmentation des études sur le Neurofeedback depuis 
les années 2000

\subsection{Principe du Neurofeedback}
Conditionnement opérant, définition phase de transfert, type de seuillage, individualisation des protocoles (iapf), 
définition rapide iapf
montage, matériel utilisé, marquage CE ou non, gestion des artefacts, définition neuromarqueur, définition impédance
définition des bandes de fréquence

\section{Les champs d'application du Neurofeedback}

\subsection{De nombreuses applications}
Epilepsie, diminution de l’anxiété, douleurs chroniques, etc

\subsection{Neurofeedback et \gls{tdah}}
Définition TDAH chez l’enfant (parler du dsm-4 et 5) et parler de l’essor de la problématique du TDAH chez l’adulte, parler des études et des méta-analyses
sham-NFB, parler des échelles cliniques

\section{Objectifs de la thèse}
Enoncé de la problématique quant à l’efficacité du NFB appliqué aux enfants TDAH
Utilisation de données cliniques pour montrer l'efficacité du NFB et proposer des améliorations du NFB appliqué aux enfants TDAH
-> étudier l'efficacité du NFB à l'aide des méta-analyses
-> identifier les facteurs ayant une influence sur l'efficacité du NFB
-> analyser un marqueur de l'attention pour mieux cibler l'entrainement par NFB

\section{Contribution et résumé des chapitres}
Lister les objectifs de la thèse :
- Réplication et mise à jour d’une méta analyse en codant les étapes en Python
- Identification des facteurs ayant une influence sur le NFB grâce à des méthodes multivariées
- Analyse de la distribution d’un marqueur de l’attention pour augmenter l’efficacité du NFB

Méthodes utilisées : méta-analyse, méthode multivariées et méthodes de partitionnement
- Résultats
- Discussion

\section{Liste des publications}