\chapter{Identification des facteurs influençant l'efficacité du Neurofeedback}

\section*{Introduction}

La réplication et la mise à jour de la méta-analyse de \citet{Cortese2016} décrite dans le chapitre précédent a permis de mettre en évidence la forte hétérogénéité des études incluses dans ce type d'analyse. 
En effet, même si ces études satisfont toutes le critère d'inclusion défini par les auteurs, elles diffèrent d'un point de vue technique et méthodologique : elles ont été rassemblées 
sans tenir compte par exemple de la qualité de l'acquisition de l'\gls{eeg}, du neuromarqueur entrainé lors du \gls{nfb} et du design de l'étude clinique (notamment le nombre de 
sessions et la durée du traitement). 

Afin de pallier ces limitations, une nouvelle approche a été implémentée : l'analyse systématique des biais (\gls{saob} en anglais) qui va justement tirer avantage de cette hétérogénéité. 
L'efficacité du traitement est considérée comme la variable dépendante expliquée par des variables indépendantes qui sont ici les facteurs méthologiques et techniques. 
Le but de cette analyse est de déterminer les facteurs qui ont une influence sur l'efficacité du \gls{nfb} et d'ainsi mettre en évidence la nécessité d'être
précautionneux dans le choix de ces paramètres. 

\section{Extraction et pré-traitement des facteurs}

La première étape de la \gls{saob} est d'obtenir les facteurs des études sélectionnées. Une liste de facteurs ayant potentiellement une influence sur l'efficacité du \gls{nfb} a été 
établie, puis les facteurs ont été extraits de chaque étude. Avant de débuter l'analyse, ils sont pré-traités en suivant les étapes décrites dans cette section. 

\subsection{Choix des facteurs} \label{choix_des_facteurs}

Les paramètres ayant une possible influence sur l'efficacité du \gls{nfb} ont été répartis en cinq catégories :
\renewcommand{\labelitemi}{$\bullet$}
\begin{itemize}
\item \emph{les biais méthodologiques :} la présence d'un groupe contrôle, l'aveugle des évaluateurs (\gls{pblind}), la randomisation des sujets dans les essais contrôlés, et la validation de l'étude 
par un \gls{irb},
\item \emph{la population :} la prise de psychostimulants durant le traitement par \gls{nfb}, la tranche d'âge des enfants inclus, la sévérité des symptômes du TDAH à pré-test (score clinique à pré-test
divisé par le score maximal à atteindre sur l'échelle clinique), et le degré d'engagement dans l'entrainement par \gls{nfb},
\item \emph{l'implémentation du \gls{nfb} :} le protocole utilisé (\gls{scp}, \gls{smr}, l'augmentation du rythme theta, l'augmentation du rythme beta dans les aires centrale ou frontale 
et la diminution du rythme theta), la présence d'une phase de transfert lors de l'entrainenement par \gls{nfb}, l'utilisation d'une carte de transfert pour s'entrainer à la maison ou à l'école, 
le type de seuillage pour les récompenses discrètes, le nombre de sessions de \gls{nfb}, la durée et la fréquence des sessions, la durée du traitement, l'individualisation des bandes de fréquence
basée sur l'\gls{iapf}, et le couplage du \gls{nfb} avec l'\gls{emg}-Biofeedback.
\item \emph{la qualité de l'acquisition :} la présence de plus d'une électrode active et la qualité de l'\gls{eeg}. Cette dernière est représentée par un indicateur allant de 1 à 3, calculé sur les critères 
suivants : 
\begin{description} 
\item[le type d'électrode utilisée :] \gls{agcl}/Gel ou \gls{au}/Gel,
\item[le contrôle de l'impédance :] la vérification du bon contact entre la peau et les électrodes en gardant l'impédance inférieure à $40$k$\Omega$,
\item[la certification du matériel hardware utilisé :] le matériel doit être conforme à la norme ISO-60601-2-26 \citep{ISO}.
\end{description}

Un score de qualité de 3 est donné si tous les critères ci-dessus sont remplis. Si au moins l'un d'eux est satisfait, le score est de 2, sinon il est mis à 1.

\item \emph{la qualité du signal} : le rejet en temps réel (l'\textit{epoch} est exclu, pas de retour calculé) ou la correction (retour calculé sur l'\textit{epoch} débruité) des 
artefacts oculaires (\gls{eog}) et le rejet en temps réel d'artefacts génériques détectés grâce à leur large amplitude. 
\end{itemize}

Afin d'éviter tout biais, le nom des facteurs a été caché durant les analyses : il n'a été révélé que lorsque le modèle a été considéré comme valide notamment au niveau 
de la normalisation des variables et de la validation des hypothèses du modèle.  

\subsection{Pré-traitement des facteurs}

Les auteurs des études incluses dans la \gls{saob} ne précisent pas systématiquement toutes les valeurs des facteurs, ce qui conduit à des observations manquantes. Afin que 
les paramètres pour lesquels peu d'observations sont diponibles ne faussent pas l'analyse, un critère d'exclusion arbitraire a été mis en place : si pour un facteur le nombre d'observations 
manquantes excède plus de 20\% du nombre total d'observations, il est exclu. Les observations manquantes dans les facteurs comportant des valeurs numériques sont 
imputées et remplacées par -1.

Par ailleurs, comme cette analyse tire avantage de l'hétérogénéité des études, si un facteur a plus de 80\% d'observations identiques, 
celui-ci est également rejeté. 

Il est important de noter qu'une étude ne correspond pas nécessairement à une observation : lorsque plusieurs échelles cliniques et/ou évaluateurs sont disponibles dans une étude,
chaque couple échelle clinique-évaluateur est considéré comme une observation.

Ensuite, les facteurs qui sont des variables catégorielles (le protocole utilisé par exemple) sont codés en \textit{dummies} : la présence du facteur est représentée par un 1 et son absence par 0. 

Enfin, les variables sont standardisées : à chaque observation est soustraite la moyenne de l'ensemble des observations, le tout divisé par l'écart-type de la moyenne de 
l'ensemble des observations. La standardisation a été choisie plutôt que la normalisation afin de garder les valeurs extrêmes dans la \gls{saob} qui auraient été mises à 0 ou 1 avec 
la normalisation. 

Les facteurs sélectionnés et prétraités sont les variables indépendantes de l'analyse.


\section{Explication de l'efficacité du Neurofeedback par des méthodes multivariées}

\subsection{Calcul de la taille d'effet intra-groupe}

L'efficacité du traitement est quantifiée par L'\gls{es}-intra-groupe. Celui-ci est calculé à partir des moyennes et écart-types des scores 
cliniques totaux donnés par les parents et les enseignants. De plus, lorsqu'une étude 
fournit des résultats pour plus d'une échelle clinique, l'\gls{es}-intra-groupe est calculé pour chaque échelle :
\begin{equation}
\label{eq:factors_effect_size_within_subject}
\text{ES-intra-groupe} = \frac{M_{\text{post},T} - M_{\text{pré},T}}{\sqrt{\frac{\sigma_{\text{pré},T}^2 + \sigma_{\text{post},T}^2}{2}}},
\end{equation} 
\noindent où $M_{\text{t},T}$ est la moyenne sur l'échelle clinique, pour le traitement $T$, au moment t (pré-test ou post-test) et $\sigma_{\text{t},T}$ représente
son écart-type. Au contraire de l'\gls{es}-inter-groupes défini à l'équation Eq.~(\ref{eq:metareview_effect_size_between}), cet \gls{es} permet de se concentrer sur l'effet du 
traitement au sein du groupe \citep{Cohen1988}. Cette définition de l'\gls{es} a déjà été précédemment utilisée dans la littérature sur le \gls{nfb} 
appliquée aux enfants \gls{tdah} \citep{Arns2009, Maurizio2014, Strehl2017}. 

Enfin, afin d'éviter de rompre les hypothèses des méthodes utilisées dans la \gls{saob}, les valeurs non comprises dans l'intervalle 
$[\mu - 3 \sigma, \mu + 3 \sigma]$ (avec $\mu$ et $\sigma$ respectivement la moyenne et l'écart-type de tous les \gls{es}-intra-groupe calculés \citep{Shewhart1931})
sont considérées comme aberrantes et sont donc rejetées.

Par la suite, l'ensemble des \gls{es}-intra-groupe est considéré comme la variable dépendante que les variables indépendantes (les facteurs) vont expliquer. 

\subsection{L'analyse systématique des biais}

La \gls{saob} comporte trois méthodes qui ont été implémentées à l'aide des bibliothèques Python Scikit-Learn \citep[version 0.18.1]{Pedregosa2011} et Statsmodels \citep[version 0.8.0]{Seabold2010} : 
\begin{itemize}
  \item une régression linéaire multiple et pondérée (\gls{wls} en anglais) \citep{Montgomery2012},
	\item une régression linéaire régularisée (\gls{lasso} en anglais) \citep{Tibshirani1996},
	\item un arbre de décision de regression (\gls{dt} en anglais) \citep{Quinlan1986}.
\end{itemize}

Les résultats de ces trois méthodes intrinséquement différentes vont être combinés : si un facteur est identifié par les trois méthodes, alors son influence sur l'efficacité du 
\gls{nfb} est plus probable que si seulement une méthode l'identifie. 

\subsubsection{La régression linéaire multiple et pondérée}
La régression linéaire a pour but d'estimer les coefficients de régression qui lient les facteurs aux \gls{es}-intra-groupe. Ici, la régression est pondérée pour, d'une part, 
prendre en compte le fait que, pour certaines études, plusieurs échelles cliniques sont disponibles, et d'autre part pour capturer les différentes tailles d'échantillon parmi les études.
Le poids $w_{i}$ associé à chaque observation $i$ est défini comme suit : 
\begin{equation}
\label{eq:weight_WLS}
w_{i} = \frac{\text{N}_{k,T}}{\text{NScales}_{k,T}},
\end{equation} 
avec $\text{N}_{k,T}$ le nombre de sujets dans l'étude $k$ dans le groupe suivant le traitement $T$ et $\text{NScales}_{k,T}$ le nombre 
d'échelles cliniques disponibles dans l'étude $k$ évaluant l'efficacité du traitement $T$.

Mathématiquement, la \gls{wls} se traduit ainsi : 
\begin{equation}
\label{eq:factors_model_WLS}
\textbf{W}y = \textbf{WX}\beta + \epsilon.
\end{equation}
$\textbf{X}$ est une matrice inversible $(n \times p)$ et représente $n$ observations sur chaque $p-1$ variable indépendante et l'intercept, 
$\beta$ est un vecteur $(p \times 1)$ des coefficients de régression associés, $\textbf{W}$ est une matrice diagonale $(n \times n)$  
des poids $w_{i}$, $y$ est un vecteur $(n \times 1)$ des variables dépendantes et $\epsilon$ est un vecteur $(n \times 1)$ d'erreurs.

Le but de la \gls{wls} est d'estimer le vecteur de coefficients $\beta$ en minimisant la somme pondérée des carrés des résidus (\gls{wrss} en anglais) :
\begin{equation}
\label{eq:factors_WRSS}
\text{WRSS} = \sum_{i=1}^{n} w_i \Big(y_i - \beta_{0} - \sum_{j=1}^{p}\beta_{j}x_{ij}\Big)^2.
\end{equation}

Une fois le vecteur $\beta$ estimé, on cherche à savoir si les hypothèses du modèle sont vérifiées : 
\begin{itemize}
	\item la matrice ${\textbf{X}}^{T}\textbf{W}^{T}\textbf{WX}$ est régulière,
  \item aucune corrélation apparente n'est trouvée entre les variables indépendantes non catégorielles, 
  \item la tendance linéaire estimée est trouvée significative en se basant sur la statistique F,
  \item les résidus sont distribués normalement en se basant sur le kurtosis et le test Omnibus.
\end{itemize} 

Si toutes ces hypothèses sont satisfaites, on peut interpréter les résultats de la \gls{wls}. On s'intéresse à la significativité de chaque coefficient $\beta_{j, 1<j<p}$ :
s'il est significativement différent de 0, le facteur associé à ce coefficient est supposé avoir une influence sur l'efficacité du \gls{nfb}. Par ailleurs, le signe du coefficient indique si 
cette influence est positive ou négative. 

Etant donné le nombre important de variables indépendantes, le pourcentage de variance estimée par la \gls{wls} est quantifié par le coefficient de 
détermination ajusté (\textit{adjusted R-Squared} en anglais) plutôt que par le coefficient de détermination simple (\textit{R-Squared} en anglais).

Une régression linéaire ordinaire (\gls{ols} en anglais) est aussi mise en place pour observer l'impact des poids sur les résultats. 

\subsubsection{La régression linéaire régularisée}

La deuxième méthode appliquée lors de la \gls{saob} est le \gls{lasso} qui intègre la sélection de variables dans le modèle linéaire grâce à la norme $\ell_1$ appliquée aux coefficients.
Les coefficients $\hat{\beta}_{j, 1<j<p}$ sont obtenus en minimisant le coût :
\begin{equation}
\label{eq:factors_lasso-minimization}
\hat{\beta} = \argmin_\beta \sum_{i=1}^{n} \Big(y_i - \beta_{0} - \sum_{j=1}^{p}\beta_{j}x_{ij}\Big)^2 + \lambda \sum_{j=1}^{p}\abs{\beta_{j}},
\end{equation} 
où $\lambda$ est le paramètre de régularisation qui, en augmentant, met de plus en plus de coefficients à 0. 

Le paramètre de régularisation optimal est déterminé par une validation croisée \textit{leave-one-out}. Cette méthode prend une seule observation 
comme donnée de test pour la validation, laissant $n$ - 1 observations pour les données d'entraînement. Le processus de la validation croisée est ensuite répété $n$ fois pour que chaque observation 
soit utilisée exactement une fois comme donnée de test. Pour chaque itération, appelée \textit{fold} en anglais, l'erreur quadratique moyenne (\gls{mse} en anglais) est calculée sur les données de test
puis les $n$ resultats sont moyennés pour mener à une seule observation qui permet de trouver le $\lambda$ optimal. Celui-ci correspond à l'abscisse du minimum de la \gls{mse} 
du \textit{fold} moyen calculée sur un large intervalle de $\lambda$ \citep{James2013}.
Un coefficient non mis à 0 signifie que le facteur associé pourrait avoir une influence sur l'efficacité du \gls{nfb} et, ici aussi, le signe du coefficient indique la direction de l'effet. 

\subsubsection{L'arbre de décision de régression}

La troisième et dernière méthode utilisée est le \gls{dt} de régression qui, à l'inverse des deux précédentes méthodes, n'est pas une méthode 
linéaire \citep{Quinlan1986}. Elle divise l'ensemble des observations en sous-ensembles de plus en plus petits en se basant sur la présence 
d'une variable qualitative ou sur la comparaison à un seuil appliqué à une variable quantitative. La position de la variable indépendante utilisée (et le
choix du seuil de comparaison dans le cas d'une variable quantitative) pour subdiviser l'ensemble des données est déterminée de façon à minimiser la
\gls{mse} définie comme suit :

\begin{equation}
\label{eq:factors_decision_tree_mse}
\text{MSE} = \frac{1}{n}\sum_{i=1}^{n} \Big(\hat{y}_i - {y}_i\Big)^2,
\end{equation}
avec $\hat{y}$ les valeurs prédites.

La première variable utilisée pour diviser l'ensemble des données se situe dans le noeud racine (\textit{root node} en anglais), les autres
variables qui mènent à une nouvelle subdivision sont dans des noeuds, et les noeuds où la divsion s'arrête sont appelés
feuilles (\textit{leaf nodes}) de l'arbre. La profondeur de l'arbre peut être définie par le nombre d'observations minimal nécessaire
pour diviser un sous ensemble. Un arbre exemple est schématisé à la Figure~\ref{Figure:factors_decision_tree_example}.

\begin{figure}[h!]
  \centering
	\includegraphics[width=1.0\linewidth]{figures/chapter-3/decision-tree-example} 
  \caption{Exemple schématique d'arbre de décision de régression. Le \textit{root node} est en rouge, les noeuds sont en bleu et les feuilles en vert. 
	Les variables A, C et D sont catégorielles ; les variables B, E et F sont numériques. Les "Valeurs\_X" correspondent à la valeur 
	prédite par l'arbre de décision de la variable dépendante suite à la division précédente.}
  \label{Figure:factors_decision_tree_example}
\end{figure}

Afin que les seuils calculés par le \gls{dt} aient un sens, la variables indépendantes n'ont ici pas été standardisées. Dans le cas de la \gls{saob}, 
les facteurs se retrouvent dans les noeuds : leur influence sur l'efficacité du \gls{nfb} est quantifiée par la valeur de la variable dépendante 
obtenue après chaque division mais aussi par leur place dans l'arbre. En effet, plus un facteur est en haut de l'arbre plus les divisions se font sur un grand nombre
d'observations, ainsi son impact sur l'efficacité est davantage probable.

\section{Analyse des facteurs influençant le Neurofeedback}

\subsection{Sélection des études}

Les termes entrés dans Pubmed pour la recherche des articles à inclure dans la \gls{saob} sont :
(ADHD OR adhd OR attention deficit disorder with hyperactivity OR minimal brain disorders OR syndrome hyperkinetic OR hyperkinetic
 syndrome OR hyperactivity disorder OR hyperactive child syndrome OR childhood hyperkinetic syndrome OR attention deficit hyperactivity disorders
 OR attention deficit hyperactivity disorder OR adhd attention deficit hyperactivity disorder OR addh OR overactive child syndrome OR attention deficit 
hyperkinetic disorder OR hyperkinetic disorder OR attention deficit disorder hyperactivity OR attention deficit disorders hyperactivity OR child 
attention deficit disorder OR hyperkinetic syndromes OR syndromes hyperkinetic OR hyperkinetic syndrome childhood) AND 
(randomized control trial OR RCT OR randomized control study OR Pilot Study OR Study OR Trial OR randomized trial) AND 
(neurofeedback OR “EEG biofeedback” OR neurotherapy OR SCP OR “slow cortical potentials” OR Theta Beta Ratio OR “TBR”). 

La dernière recherche effectuée le 12 février 2018 avec ces termes a retourné 155 résultats, auxquels se sont ajoutés 22 articles inclus dans les précédentes 
méta-analyses sur le \gls{nfb} appliqué aux enfants \gls{tdah} \citep{Arns2009, Sonuga-Barke2013, Micoulaud2014, Cortese2016, Catala2017}. Afin de sélectionner
les études à inclure dans la \gls{saob}, les 177 résultats ont été filtrés à l'aide du pipeline représenté à la 
Figure~\ref{Figure:factors_pipeline_selection_studies}. Au final $k$ = 33 études ont été retenues, qui correspondent par ailleurs au critère d'inclusion de
\citet{Cortese2016} sans les exigences sur les groupes contrôles. 

\newpage\
\begin{figure}[h!]
  \centering
	\includegraphics[width=1.0\linewidth]{figures/chapter-3/factors-selection-studies} 
  \caption{Diagramme de sélection des études pour l'analyse systématique des biais (dernière recherche le 12 fevrier 2018).}
  \label{Figure:factors_pipeline_selection_studies}
\end{figure}

Les \gls{es}-intra-groupe sont calculés pour chaque étude puis les valeurs aberrantes sont rejetées. La distribution des \gls{es}-intra-groupe ainsi que les bornes 
de l'intervalle d'inclusion sont représentées à la Figure~\ref{Figure:distribution_ES_within}. Les \gls{es}-intra-groupe négatifs sont en faveur du \gls{nfb}.

\begin{figure}[h!]
  \centering
	\includegraphics[width=0.7\linewidth]{figures/chapter-3/distribution-ES-within} 
  \caption{Distribution des tailles d'effet (\gls{es}) intra-groupe, une valeur négative est en faveur du Neurofeedback. Les lignes verticales noires correspondent aux bornes supérieure et 
	inférieure de l'intervalle dans lequel les observations sont acceptées.}
  \label{Figure:distribution_ES_within}
\end{figure}

Deux observations sont à l'extérieur de l'intervalle d'inclusion : deux groupes de sujets de \citet{Bazanova2018} (celui de l'individualisation du \gls{nfb} et celui de
l'individualisation du \gls{nfb} et du couplage avec \gls{emg}-Biofeedback). En effet, ces deux groupes présentent des \gls{es}-intra-groupe très larges (respectivement -3.41 et -3.95)
qui sont même plus élevés que ceux rapportés dans la littérature sur l'efficacité des psychostimulants sur les symptômes du \gls{tdah} chez les enfants \gls{tdah} \citep{Luan2017}.
Ces valeurs invalident nos hypothèses de travail, ainsi afin de pouvoir conclure sur les résultats obtenus par la \gls{saob}, un rejet des valeurs aberrantes a été implémenté.

La \gls{saob} est donc effectuée sur 33 études (qui correspondent à 67 observations) évaluant l'efficacité du \gls{nfb} sur les enfants \gls{tdah} et qui son listées dans le 
Table~\ref{Table:table_factors_analysis_meta_analysis_list_studies}. Au total, les 33 études sélectionnées rassemblent 846 enfants \gls{tdah} effectuant du \gls{nfb}.

\newpage\
\begin{table}[h!]
  \centering
  \caption{Liste des études incluses dans l'analyse systématique des biais : a) études incluses dans \citet{Cortese2016}
	(dernière recherche le 30 août 2015) ; b) études satisfaisant le critère d'inclusion de \citet{Cortese2016} (dernière recherche le 12 février 2018) ; c) études 
	satisfaisant le critère d'inclusion de \citet{Cortese2016} à l'exception de la partie concernant le groupe contrôle (dernière recherche le 12 février 2018).}
  \fontsize{9}{11}\selectfont
\begin{tabular}{ cccccc }
\toprule
\multicolumn{3}{ c }{Analyse} & Etude & Année & \shortstack{ Nombre de sujets \\ dans le groupe \\ \gls{nfb} } \\
\midrule
 & & & \citeauthor{Arnold2014} & 2014 & 26 \\ 
 & & & \citeauthor{Bakhshayesh2011} & 2011 & 18 \\
 & & & \citeauthor{Beauregard2006} & 2006 & 15 \\
 & & & \citeauthor{Bink2014} & 2014 & 45 \\
 & & & \citeauthor{Christiansen2014} & 2014 & 14 \\
 & & & \citeauthor{Gevensleben2009} & 2009 & 59 \\
 & & & \citeauthor{Heinrich2004} & 2004 & 13 \\
 & & & \citeauthor{Holtmann2009} & 2009 & 20 \\
 & & & \citeauthor{Linden1996} & 1996 & 9 \\
 & & & \citeauthor{Maurizio2014} & 2014 & 13 \\
 & & & \citeauthor{Steiner2011} & 2011 & 9 \\
 & & & \citeauthor{Steiner2014} & 2014 & 34 \\
 & & & \citeauthor{VanDongen2013} & 2013 & 22 \\
 & & \shortstack{a = Réplication de \\ \citeauthor{Cortese2016}  \\ (voir \ref{replication}) } & \textbf{13 études} & & \textbf{297} \\
\cmidrule(lr){3-6}
 & & & \citeauthor{Aggensteiner2019} & 2019 & 75 \\
 & & & \citeauthor{Baumeister2016} & 2016 & 8 \\
 & & & \citeauthor{Bazanova2018} & 2018 & 17 \\
 & & & \citeauthor{Minder2018} & 2018 & 38 \\
 & & & \citeauthor{Moreno2019} & 2019 & 19 \\
 & & & \citeauthor{Strehl2017} & 2017 & 72 \\
 & & & \citeauthor{Shereena2019} & 2019 & 15 \\
 & \shortstack{b = Mise à jour \\ \citeauthor{Cortese2016} \\ (voir \ref{selection_studies}) } & & \textbf{16 études} & & \textbf{541} \\
\cmidrule(lr){2-6}
 & & & \citeauthor{Bluschke2016} & 2016 & 19 \\
 & & & \citeauthor{Cueli2019} & 2019 & 64 \\
 & & & \citeauthor{Deilami2016} & 2016 & 12 \\
 & & & \citeauthor{Drechsler2007} & 2007 & 17 \\
 & & & \citeauthor{Duric2012} & 2012 & 23 \\
 & & & \citeauthor{Escolano2014} & 2014 & 20 \\
 & & & \citeauthor{Fuchs2003} & 2003 & 22 \\
 & & & \citeauthor{Gelade2016} & 2016 & 39 \\
 & & & \citeauthor{Heinrich2019} & 2019 & 60 \\
 & & & \citeauthor{Kropotov2005} & 2005 & 86 \\
 & & & \citeauthor{Lee2017} & 2017 & 18 \\
 & & & \citeauthor{Leins2007} & 2007 & 19 \\
 & & & \citeauthor{Li2013} & 2013 & 32 \\
 & & & \citeauthor{Meisel2014} & 2014 & 12 \\
 & & & \citeauthor{Mohagheghi2017} & 2017 & 30 \\
 & & & \citeauthor{Mohammadi2015} & 2015 & 16 \\
 & & & \citeauthor{Monastra2002} & 2002 & 51 \\
 & & & \citeauthor{Ogrim2013} & 2013 & 13 \\
 & & & \citeauthor{Rajabi2019} & 2019 & 16 \\
 & & & \citeauthor{Sudnawa2018} & 2018 & 20 \\
 & & & \citeauthor{Strehl2006} & 2006 & 23 \\
 c = \gls{saob} & & & \textbf{41 études} & & \textbf{1 153} \\
\bottomrule
\end{tabular}

  \label{Table:table_factors_analysis_meta_analysis_list_studies}
\end{table}

\newpage\
\subsection{Facteurs identifiés}

Vingt-huit paramètres ont été initialement identifiés afin d'analyser leur influence sur l'efficacité du \gls{nfb}. Parmi eux, neuf ont dû être exclus car ils étaient trop
homogènes ou présentaient trop d'observations manquantes : 
\begin{itemize}
	\item l'augmentation du rythme beta dans les aires frontales,
  \item l'utilisation d'une carte pour le transfert de l'entrainement à la maison et à l'école, 
  \item le type de seuillage pour les récompenses discrètes (incrémental ou fixe),
  \item la qualité de l'acquisition de l'\gls{eeg} égale à 3,
	\item la présence d'un groupe contrôle,
	\item l'individualisation des bandes de fréquences basée sur la valeur de l'\gls{iapf},
	\item le couplage entre le \gls{nfb} et l'\gls{emg}-Biofeedback,
	\item la sévérité des symptômes du \gls{tdah},
	\item le degré d'engagement dans l'entraînement par \gls{nfb}.
\end{itemize} 


Afin de comparer la variabilité des valeurs au sein des facteurs non catégoriels sélectionnés, les boxplots des valeurs standardisées ont été obtenus et représentés à la Figure~\ref{Figure:factors-boxplots} :
\begin{figure}[h!]
  \centering
	\includegraphics[width=0.7\linewidth]{figures/chapter-3/factors-distribution-of-factors} 
  \caption{Boxplots des facteurs aux valeurs numériques standardisées.} 
  \label{Figure:factors-boxplots}
\end{figure}

Le nombre de sessions et la durée de la session sont plus variables que l'âge minimal et maximal des enfants inclus dans l'étude. Les extremum et la moyenne de ces facteurs aux valeurs numériques
non standardisées sont listés dans la Table~\ref{Table:table_factors_extremum_values} pour donner une idée de la variabilité de ces facteurs dans les études incluses. 

\begin{table}[h!]
  \centering
  \caption{Extremum et moyenne des facteurs numériques non standardisés.}
  \fontsize{9}{11}\selectfont
\begin{tabular}{ cccc }
\toprule
Facteurs & Valeur minimale & Valeur maximale & Moyenne ($\pm$ std) \\
\midrule
Age max (en années) & 24 & 8.33 & 13.15 (3.16) \\
Age min (en années) & 5 & 12.33 & 8.00 (1.96) \\
Fréquence des sessions (par semaine) & 7 & 1 & 2.66 (1.37) \\
Nombre de sessions & 10 & 43 & 29.96 (8.78) \\
Durée de la session (en minutes) & 90 & 16 & 43.23 (22.42) \\
Durée du traitement (en semaines) & 30 & 3.0 & 13.17 (6.65) \\
\bottomrule
\end{tabular}
  \label{Table:table_factors_extremum_values}
\end{table}

Les trois méthodes décrites précédemment sont donc appliquées tour à tour sur les 19 facteurs restants. Tous les résultats sont résumés dans la Table~\ref{Table:table_factors_analysis_results_summary}.

\newpage\
\begin{table}[h!]
  \centering
  \caption{Resultats de la régression linéaire pondérée (\gls{wls}), de la régression linéaire régularisée (\gls{lasso}) et de l'arbre de décision (\gls{dt}). Pour la \gls{wls}, une p-value $<$ 0.05 
	(en gras) signifie que le coefficient du facteur correspondant est significativement différent de 0. Pour le \gls{lasso}, les facteurs dont les coefficients sont non mis à 0 (en gras) sont 
	sélectionnés. Pour l'arbre de décision, la place du facteur dans l'arbre est indiquée. Pour les deux premières colonnes, quand la valeur du coefficient est négative le facteur 
	correspondant pourrait mener à de meilleurs résultats du \gls{nfb}.}
  \begin{center}
\small 
\begin{tabular}{ p{3cm} p{3cm} p{3cm} p{2cm} p{2cm} p{2cm}}
\toprule
\multicolumn{2}{c}{ \shortstack{Variables \\ indépendantes (facteurs)} } & \shortstack{ Coefficients \\ trouvés par \gls{wls} \\ ($p$-value) } & \shortstack{ Coefficients \\ trouvés par \\ \gls{lasso} } & \shortstack{Place \\ sur le \\ \gls{dt}} \\
\midrule
\multirow{ 3}{*}{ \textit{Méthodologiques} } & \gls{pblind} & \hskip 0.12in\textbf{0.15 (0.015)} & \hskip 0.12in\textbf{0.086} & \textbf{\textit{root node}} \\ 
& randomisation & \hskip 0.12in0.013 (0.840) & \hskip 0.12in0.00 & / \\   
\midrule
\multirow{ 3}{*}{ \textit{Population} } & age max & \hskip 0.08in-0.10 (0.106) & \hskip 0.12in0.00 & / \\
& age min & \hskip 0.12in0.056 (0.43) & \hskip 0.12in0.00 & / \\
& prise de médicaments & \hskip 0.08in-0.026 (0.72) & \hskip 0.12in0.00 & / \\
\midrule
\multirow{ 9}{*}{ \textit{ \shortstack{Implementation \\ du \gls{nfb}} } } & nombre de sessions & \hskip 0.08in\textbf{-0.19 (0.025)} & \hskip 0.12in0.00 & / \\
& durée de la session & \hskip 0.08in-0.13 (0.16) & \hskip 0.12in0.00 & \textbf{2$^{eme}$ noeud} \\
& durée du traitement & \hskip 0.12in\textbf{0.39 (0.00)} & \hskip 0.12in\textbf{0.098} & \textbf{2$^{eme}$ et 3$^{eme}$ noeuds} \\
& fréquence des sessions & \hskip 0.12in0.027 (0.690) & \hskip 0.08in\textbf{-0.055} & \textbf{2$^{eme}$ noeud} \\ 
& \gls{smr} & \hskip 0.12in0.12 (0.067) & \hskip 0.12in0.00 & / \\
& augmentation de beta en central & \hskip 0.12in0.087 (0.32) & \hskip 0.12in\textbf{0.026} & / \\  
& diminution de theta & \hskip 0.08in-0.095 (0.39) & \hskip 0.12in0.00 & / \\
& \gls{scp} & \hskip 0.08in-0.14 (0.30) & \hskip 0.12in0.00 & / \\ 
& phase de transfert & \hskip 0.12in\textbf{0.33 (0.001)} & \hskip 0.12in\textbf{0.079} & \textbf{1$^{er}$ noeud} \\
\midrule
\multirow{ 2}{*}{ \textit{ \shortstack{Qualité de \\ l'acquisition} } } & plus d'une électrode d'enregistrement & \hskip 0.08in-0.083 (0.20) & \hskip 0.08in\textbf{-0.030} & / \\ 
& \gls{eeg} qualité 2 & \hskip 0.08in\textbf{-0.28 (0.00)} & \hskip 0.08in\textbf{-0.037} & / \\  
\midrule
\multirow{ 2}{*}{ \textit{Qualité du signal} } & rejet ou correction des artefacts oculaires & \hskip 0.08in-0.10 (0.164) & \hskip 0.08in\textbf{-0.0046} & \textbf{1$^{er}$ noeud} \\ 
& rejet des artefacts basé sur l'amplitude & \hskip 0.12in0.14 (0.058) & \hskip 0.12in0.00 & / \\   
\bottomrule
\end{tabular}
\end{center}

  \label{Table:table_factors_analysis_results_summary}
\end{table}

\subsubsection{La régression linéaire multiple et pondérée}

Les hypothèses de ce modèle sont respectées, les résultats sont interprétables :
\begin{itemize}
	\item la matrice ${\textbf{X}}^{T}\textbf{W}^{T}\textbf{WX}$ est bien régulière,
  \item aucune corrélation apparente n'est effectivement trouvée entre les variables indépendantes non catégorielles, 
  \item la tendance linéaire estimée est trouvée significative (Prob(F-statistic) = 7.58e-08),
  \item les résidus sont distribués normalement (kurtosis = 3.154 et Prob(Omnibus) = 0.392).
\end{itemize}

La \gls{wls} a trouvé 9 facteurs significatifs (deuxième colonne de la Table~\ref{Table:table_factors_analysis_results_summary}) avec un \textit{adjusted R Squared} de 0.62. 
Dans le cas de l'\gls{ols}, les mêmes facteurs ont été trouvés significatifs (à l'exception de la qualité de 
l'\gls{eeg} égal à 2 et de la présence de plus d'une électrode active) mais avec un \textit{adjusted R Squared} plus faible (0.35). Ainsi, associer un poids
à chaque observation permet d'expliquer davantage de variabilité. 

Il est important de noter qu'étant donné qu'un \gls{es}-intra-groupe négatif est en faveur de l'efficacité du \gls{nfb},
un facteur dont le coefficient est négatif aurait une influence positive sur les résultats \gls{nfb}.


\subsubsection{La régression linéaire régularisée}

La validation croisée \textit{leave-one-out} illustrée à la Figure~\ref{Figure:selection_lambda_lasso} a permis de déterminer un $\lambda$ optimal égal à 0.059.
\begin{figure}[h!]
  \centering
	\includegraphics[width=0.7\linewidth]{figures/chapter-3/factors-selection-lasso-best-lambda} 
  \caption{Erreur quadratique moyenne (\gls{mse}) obtenue sur la moyenne de tous les \textit{folds} utilisés lors de la validation 
	croisée \textit{leave-one-out}. La courbe bleue représente la \gls{mse} moyennée sur tous les \textit{folds} ; la droite verticale en pointillé correspond
	au minimum de la \gls{mse} moyenne.}
  \label{Figure:selection_lambda_lasso}
\end{figure}

Le \gls{lasso} a gardé six facteurs différents de 0 (troisième colonne de la Table~\ref{Table:table_factors_analysis_results_summary}). Pour cette méthode également,
un facteur dont le coefficient est négatif aurait un bon impact sur l'efficacité du \gls{nfb}.

\subsubsection{L'arbre de décision de régression}

L'arbre de décision obtenu est présenté à la Figure~\ref{Figure:factors_decision_tree} : \gls{pblind} est le meilleur prédicteur (dernière colonne de la
Table~\ref{Table:table_factors_analysis_results_summary}). Quatre autres facteurs divisent ensuite les sous-ensembles, toutefois étant donné que de moins
en moins d'observations sont disponibles plus on descend dans l'arbre, l'influence de ces facteurs est de moins en moins certaine.

\begin{figure}[h!]
  \centering
	\includegraphics[width=1\linewidth]{figures/chapter-3/factors-decision-tree} 
  \caption{Arbre de décision obtenu : \gls{es} correspond à l'\gls{es}-intra-groupe et $k$ au nombre d'études. L'importance des variables indépendantes
	décroît depuis le \textit{root node}. La durée de la session est mesurée en minutes, la durée du traitement en semaines et l'âge en années.}
  \label{Figure:factors_decision_tree}
\end{figure}

\subsubsection{Résumé des résultats}

Plusieurs facteurs ont été trouvés significatifs par plus d'une méthode, notamment les évaluateurs \gls{pblind}, la qualité de l'\gls{eeg} égale à 2 et la durée du traitement. 
De plus, dans le cas de ces trois facteurs, toutes les méthodes s'accordent quant à la direction de leur influence : alors qu'évaluer l'efficacité du \gls{nfb} en étant \gls{pblind}
semble conduire à de moins bons résultats, une durée de traitement plus courte et l'acquisition de l'\gls{eeg} avec un bon matériel mèneraient à un traitement plus efficace. 

L'influence des facteurs retournés par une ou deux méthodes est plus incertaine :
\begin{itemize}
	\item la \gls{wls} et le \gls{lasso} trouvent tous deux qu'utiliser plus d'une électrode active durant la session de \gls{nfb} mènerait à de bons résultats,
  \item la \gls{wls} et le \gls{dt} conluent tous deux qu'effectuer un grand nombre de sessions est préférable, 
  \item le \gls{lasso} et le \gls{dt} obtiennent tous deux qu'un nombre élevé de sessions par semaine influencerait positivement les résultats. 
\end{itemize}

Cinq facteurs sont retournés seulement par une méthode : la randomisation des groupes, le recours à un \gls{irb}, la durée de la session, la présence d'une phase de transfert,
et la correction ou rejet des artefacts oculaires. Huit facteurs, n'ont quant à eux été sélectionnés par aucune méthode : l'âge minimum et maximum des enfants, prendre des 
médicaments pendant le traitement par \gls{nfb}, les protocoles \gls{smr}, augmentation de beta dans les aires centrales, diminution de theta et \gls{scp}, et la correction
des artefacts basée sur l'amplitude. Ainsi ces facteurs n'influenceraient pas l'efficacité du \gls{nfb}. 

\section{Discussion}

La description et l'analyse des différents types d'implémentation de \gls{nfb} ont fait l'objet de plusieurs études \citep{Arns2014, 
Jeunet2018, Arns2009, Cortese2016}. Cependant, à notre connaissance, aucune de ces études n'a implémenté 
une approche systématique et multivariée pour associer les facteurs aux résultats cliniques.

\subsection{Facteurs et efficacité du \gls{nfb}}

Plusieurs facteurs ont été identifiés par deux méthodes et même si la confiance accordée à leur influence sur l'efficacité du \gls{nfb} est moindre par rapport
à celle donnée aux facteurs retournés par trois méthodes, il est tout de même intéressant de se pencher sur eux, notamment sur un facteur qui 
parait intuitivement important : le nombre de sessions.

Comme attendu, le nombre de sessions s'avère être identifié par deux méthodes ce qui est en accord avec la littérature existante. En effet, grâce à plusieurs régressions linéaires 
simples sans correction pour tests multiples \citep{Arns2009}, \citet{Arns2014} affirme qu'effectuer moins de 20 sessions de \gls{nfb} conduit à une moins bonne efficacité. De même,
\citet{Vernon2004} a observé que des changements positifs aussi bien sur l'\gls{eeg} qu'au niveau comportemental apparaissent après au moins 20 sessions. Toutefois, \citep{Enriquez2017}
souligne le fait que le nombre de sessions doit être choisi avec précaution afin d'éviter "l'\textit{overtraining}". Le fait que le nombre de sessions n'a pas été identifié par le \gls{lasso}
pourrait s'expliquer par la présence de seulement deux observations de 20 sessions ou moins. Ainsi, étant donné que le seuil minimal pour obtenir des résultats avec l'entrainement
par \gls{nfb} semble dépassé pour la très grande majorité des observations, il est donc peu probable que ce facteur puisse être retourné par les trois méthodes sur cet ensemble de données. 
Toutefois, les deux méthodes qui ont identifié ce facteur s'accordent toutes deux sur la direction de l'effet : comme attendu plus le nombre de sessions effectué est important, 
plus le \gls{nfb} semble être efficace. 

Le type de protocole \gls{nfb} n'a été identifié par aucune méthode et donc n'influencerait pas l'efficacité du \gls{nfb}. Cette importance minime octroyée par la \gls{saob} au type de protocole
est contre-intuitive étant donné le rôle central du protocole choisi sur le mode d'action neurophysiologique et donc sur l'impact sur l'efficacité thérapeutique \citep{Vernon2004}. Une probable 
explication pour ce résultat est que tous ces protocoles ont une efficacité équivalente sur les populations étudiées et donc ne représentent pas un facteur explicatif significatif. 
Toutefois, ce résultat n'exclut pas une stratégie personnalisée basée sur les phénotypes pour augmenter les performances, comme précédemment suggéré par \citet{Alkoby2017}.

Trois facteurs ont été identifiés par les trois méthodes avec, de plus, la même direction d'influence : la qualité de l'\gls{eeg} égale à 2, la durée du traitement, et les évaluateurs 
probablement aveugles au traitement. 

Tout d'abord, la \gls{saob} a montré qu'enregistrer l'\gls{eeg} dans de bonnes conditions mène à de meilleurs résultats. Cette observation peut s'expliquer par le fait qu'un signal
\gls{eeg} de bonne qualité permet l'extraction plus précise des caractéristiques de l'\gls{eeg} liées au \gls{tdah} et donc conduit à un meilleur apprentissage et 
à une efficacité thérapeutique augmentée. Cependant, évaluer la qualité des moyens d'acquisition de l'\gls{eeg} (comme l'amplificateur utilisé) 
est difficile du fait du peu d'informations fourni par les études à ce sujet. Par conséquent, les futures essais cliniques devraient apporter plus 
de précisions quant au matériel utilisé afin de pouvoir plus aisément juger de sa qualité. 

Ensuite, il semblerait que plus le traitement par \gls{nfb} est long, moins il devient efficace. Le degré d'engagement dans l'intervention pourrait expliquer ce
résultat : être engagé dans un traitement long est plus compliqué. Cependant, il est difficile de quantifier ce degré car soit aucun questionnaire n'est rempli 
par les enfants à ce sujet, soit cette information n'est pas mentionnée. 

La durée du traitement est étroitement liée à son intensité, ainsi il est possible de supposer qu'une période de traitement plus courte est préférable 
car la fréquence des sessions est plus élevée. Cette hypoyhèse est étayée par 
le fait que la variable "fréquence des sessions" (nombre de sessions par semaine) est aussi associée à de plus grands \gls{es}-intra-groupe selon le
\gls{lasso} et le \gls{dt}. L'impact de l'intensité du traitement par \gls{nfb} a été exploré par \citet{Rogala2016} sur des sujets adultes sains : il est
observé que les études proposant au moins 4 sessions de \gls{nfb} sur des jours consécutifs sont toutes bénéfiques. Ainsi, ces résultats indiquent qu'adopter une fréquence 
élevée de sessions est préférable, ce qui est peu connu dans le domaine du \gls{nfb}.

\subsection{Perspectives pour de futures analyses}

L'influence d'autres facteurs sur l'efficacité du \gls{nfb} aurait été intéressante à analyser, comme la personnalisation des protocoles d'entrainement basée sur l'\gls{iapf},
dont les résultats paraissent prometteurs selon \citet{Bazanova2018} et \citet{Escolano2014}. Cependant elle n'a pas pu être incluse dans la \gls{saob} faute d'un 
nombre suffisant d'études proposant un protocole personnalisé. Ce manque d'études est aussi la raison pour laquelle le couplage entre l'\gls{emg}-Biofeedback et le 
\gls{nfb} n'a pas pu être étudié dans la \gls{saob}. 

Un autre facteur intéressant, qui aurait pu aider à expliquer les résultats sur la durée du traitement, a 
également été exclu de l'analyse : la sévérité des symptômes à pré-test. Bien que les scores à pré-test soient disponibles pour chaque étude, ils ne sont pas
comparables car différentes échelles sont utilisées. Afin de résoudre ce problème, ces scores ont été normalisés grâce au score maximum pouvant être atteint
sur chaque échelle. Toutefois, cette valeur n'a pas pu être trouvée pour plusieurs échelles cliniques : trop d'observations manquantes ont mené au rejet de facteur.

Enfin, il serait intéressant de s'intéresser au lieu où les sessions de \gls{nfb} sont effectuées. En effet, \citet{Minder2018} souligne le fait que le lieu
d'entrainement pourrait aussi être un facteur contribuant à l'efficacité du \gls{nfb}. Toutefois, dans son étude aucune différence significative n'est trouvée entre
les résultats des performances à la clinique et à l'école. Dans la grande majorité des études sur le \gls{nfb} appliqué aux enfants
\gls{tdah}, les sessions ont lieu en clinique, rendant impossible à la \gls{saob} d'étudier ce facteur. Cependant, à l'instar de \citet{Minder2018}, d'autres 
études effectuent leurs séances en dehors du milieu hospitalier \citep{Bioulac2019}, ainsi l'influence du lieu d'entrainement pourra finir par être étudiée
plus précisément. Cela sera aussi valable pour les facteurs évoqués plus tôt et qui ont dû être exclus de la \gls{saob}.

\subsection{Analyse approfondie des évaluateurs probablement aveugles}

Les résultats de la \gls{saob} sont globalement en faveur de l'efficacité du \gls{nfb} pour traiter les enfants \gls{tdah}. Toutefois, comme attendu, l'évaluation
des symptômes par des personnes non-aveugles (\gls{mprox}) conduit à des résultats plus favorables que celles des évaluateurs \gls{pblind}, ce qui est en accord avec les
méta-analyses existantes \citep{Micoulaud2014, Cortese2016}. 

Les enseignants sont considérés comme \gls{pblind} par \citet{Cortese2016, Micoulaud2014} et cette définition a été suivie dans la \gls{saob}. Etonnamment, les données
à disposition ne vont pas exactement dans le sens de l'hypothèse largement acceptée que la différence entre \gls{mprox} et \gls{pblind} peut être seulement 
expliquée par l'effet placebo. Par ailleurs, les enseignants sont désignés comme "probablement" aveugles, ce qui signifie qu'ils peuvent être au courant du
traitement suivi par les enfants. 

Un élément corrobore cette hypothèse : pour chaque étude incluse dans cette analyse, l'évaluation des symptômes avant le début
du traitement par les parents comparée à celle des enseignants montre que ces derniers ne voient pas l'ensemble des symptômes, autrement dit qu'ils sont probablement
plus aveugles aux symptômes qu'au traitement comme l'illustre la Figure~\ref{Figure:factors_pblind_discussion}. En effet, à pré-test, les enseignants évaluent les 
symptômes moins sévèrement que les parents et observent moins d'amélioration à post-test : cela correspond davantage au cas \textbf{A} représentant l'absence d'effet placebo
qu'au cas \textbf{B}.

\begin{figure}[h!]
  \centering
	\includegraphics[width=1\linewidth]{figures/chapter-3/factors-pblind-discussion} 
  \caption{Scores à pré-test et post-test ($\pm$ erreur type) donnés par les parents (\gls{mprox}) en bleu et les enseignants (\gls{pblind}) en pointillés verts. 
	Deux hypothèses sur des données hypothétiques : \textbf{(A)} pas d'effet placebo, les enseignants notent moins de symptômes du \gls{tdah} donc la difference entre pré et post-test est faible
	et \textbf{(B)} effet placebo, les enseignants observent autant de symptômes à pré-test que les parents mais pas autant d'amélioration. \textbf{(C)} Données réelles : 
	évolution des scores attribués par les parents et enseignants entre pré- et post-test dans les études qui satisfont le critère d'inclusion de \citeauthor{Cortese2016} 
	et qui donnent les scores sur les mêmes échelles pour les deux types d'évaluateurs.}
  \label{Figure:factors_pblind_discussion}
\end{figure}

Ces différences d'évaluation entre parents et enseignants ont été étudiées à de multiples reprises \citep{Sollie2013, Narad2015, Minder2018}, montrant que 
ces derniers sont plus susceptibles de sous-estimer la sévérité des symptômes du \gls{tdah} chez l'enfant, surtout chez les plus jeunes. Par conséquent, les 
enseignants sont peut-être simplement moins capables d'observer un changement clinique durant la durée du traitement. Par ailleurs, les scores entre enseignants 
sont plus variables que ceux entre parents, ce qui peut en partie expliquer le faible \gls{es} (aussi bien intra que inter-groupes) calculé pour les évaluateurs
\gls{pblind}. Ainsi pour conclure, recourir aux évaluateurs \gls{pblind} pour estimer l'effet placebo n'apparait pas comme étant un choix approprié. 

Une autre façon de mettre en évidence un éventuel effet placebo est de se rapporter au \gls{dt} présenté à la Figure~\ref{Figure:factors_decision_tree}. Le 
\textit{root node} divise l'ensemble des données en deux parties : d'une part un sous arbre est créé avec 46 observations correspondant aux évaluateurs
\gls{mprox} et d'autre part, 21 observations correspondant aux évaluateurs \gls{pblind}. Si les différences observées entre ces deux types d'évaluateurs est due à l'effet
placebo, il serait attendu que le sous-arbre des observations \gls{mprox} comporte des facteurs liés à la perception de l'implication dans le traitement. En effet, dans cette
partie de l'arbre on trouve bien le facteur "`durée du traitement", mais qui ne va pas dans la direction corroborant l'effet placebo : intuitivement, 
il est attendu que plus le traitement est long, plus l'effet placebo est important et plus l'\gls{es}-intra-groupe est élevé, or ici l'inverse est observé ce qui
contredit l'hypothèse. 

Ainsi, ces résultats suggèrent que les évaluateurs \gls{pblind} peuvent difficilement être utilisés pour quantifier l'effet placebo étant donné qu'ils paraissent
plus aveugles aux symptômes qu'au traitement. En l'absence de protocoles de \textit{sham}-\gls{nfb} éthiquement \citep{Holtmann2014} et techniquement \citep{Birbaumer1991} 
faisables, il est nécessaire d'avoir recours à une méthode alternative acceptable pour juger de l'efficacité du traitement \citep{World-Medical-Association2000}. Une analyse robuste envisageable 
serait d'étudier les neuromarqueurs collectés durant le traitement par \gls{nfb} pour démontrer que les patients contrôlent effectivement le neuromarqueur, qu'ils 
apprennent (c'est à dire que le contrôle du neuromarqueur s'intensifie avec le temps), et que cet apprentissage conduit à une réogarnisation cérébrale durable. 

\subsection{Validation prospective sur les nouvelles études publiées depuis le 18/02/2018}

L'analyse précédente présente des résultats obtenus sur des études publiées avant le 12 février 2018 \citep{Bussalb2019a}, or la recherche dans le domaine du \gls{nfb} appliqué aux enfants \gls{tdah} est
en pleine expansion conduisant à de nombreuses publications en peu de temps.

\subsubsection{Sélection des nouvelles études} 

Les résultats décrits précédemment ont été obtenus avec les études disponibles sur PubMed le 12 février 2018. Cependant, depuis cette date d'autres études satisfaisant
les critères d'inclusion présentés à la Figure~\ref{Figure:factors_pipeline_selection_studies} ont été publiées. Ainsi, la \gls{saob} a été mise à jour avec ces 
nouvelles études identifiées au 2 septembre 2019 dont les principales caractéristiques sont présentées dans la Table~\ref{Table:factors_new_studies_update}.

\begin{table}[h!]
  \centering
  \caption{Liste des études incluses dans l'analyse systématique des biais mise à jour : a) études satisfaisant le critère d'inclusion de \citet{Cortese2016} (dernière recherche 
	le 2 septembre 2019) ; c) études satisfaisant le critère d'inclusion de \citet{Cortese2016} à l'exception de la partie concernant le groupe contrôle 
	(dernière recherche le 2 septembre 2019).}
  \fontsize{9}{11}\selectfont
\begin{tabular}{ cccccc }
\toprule
\multicolumn{3}{ c }{Analyse} & Etude & Année & \shortstack{ Nombre de sujets \\ dans le groupe \\ Neurofeedback } \\
\midrule
 & & & \citeauthor{Aggensteiner2019} & 2019 & 75 \\
 & & & \citeauthor{Minder2018} & 2018 & 38 \\
 & & & \citeauthor{Moreno2019} & 2019 & 19 \\
 & & & \citeauthor{Shereena2019} & 2019 & 15 \\
 & \shortstack{a = Mise à jour \\ \citeauthor{Cortese2016} \\ (voir \ref{replication_and_update}) } & & 4 études & & 147 \\
\cmidrule(lr){2-6}
 & & & \citeauthor{Heinrich2019} & 2019 & 60 \\
 & & & \citeauthor{Rajabi2019} & 2019 & 16 \\
 & & & \citeauthor{Cueli2019} & 2019 & 64 \\
 & & & \citeauthor{Sudnawa2018} & 2018 & 20 \\
 b = Analyse \\ Systématique des \\bias & & & 8 études & & 160 \\
\bottomrule
\end{tabular}

  \label{Table:factors_new_studies_update}
\end{table}

Cette mise à jour permet ainsi d'appliquer la \gls{saob} sur 41 études ce qui correspond à 84 observations, augmentant ainsi la robustesse des
résultats. Malheureusement, tous les facteurs exclus dans l'analyse antérieure le sont également avec cette mise à jour et la validation par
un \gls{irb} est également rejetée.

\subsubsection{Résumé des résultats} 

Les hypothèses des méthodes utilisées dans la \gls{saob} sont validées, il est donc possible d'analyser les résultats qui sont présentés dans la 
Table~\ref{Table:table_factors_analysis_results_summary_update}.

\begin{table}[h!]
  \centering
  \caption{Resultats des mises à jour de la régression linéaire pondérée (\gls{wls}), de la régression linéaire régularisée (\gls{lasso}) et de l'arbre de décision (\gls{dt}). Pour la \gls{wls}, une p-value $<$ 0.05 
	(en gras) signifie que le coefficient du facteur correspondant est significativement différent de 0. Pour le \gls{lasso}, les facteurs dont les coefficients sont non mis à 0 (en gras) sont 
	sélectionnés. Pour l'arbre de décision, la place du facteur dans l'arbre est indiquée. Pour les deux premières colonnes, quand la valeur du coefficient est négative le facteur 
	correspondant pourrait mener à de meilleurs résultats du \gls{nfb}. Les valeurs en vert correspondent aux valeurs devenues significatives après la mise à jour ; les valeurs
	en rouges correspondent aux valeurs ayant perdu la significativité après la mise à jour.}
  \begin{center}
\small 
\begin{tabular}{ p{3cm} p{3cm} p{3cm} p{2cm} p{2cm} p{2cm}}
\toprule
\multicolumn{2}{c}{ \shortstack{Variables \\ indépendantes (facteurs)} } & \shortstack{ Coefficients \\ trouvés par \gls{wls} \\ (p-value) } & \shortstack{ Coefficients \\ trouvés par \\ \gls{lasso} } & \shortstack{Place \\ sur le \\ \gls{dt}} \\
\midrule
\multirow{ 3}{*}{ \textit{Méthodologiques} } & \gls{pblind} & \hskip 0.12in\textbf{0.16 (0.023)} & \hskip 0.12in\textbf{0.045} & \textbf{\textit{root node}} \\ 
& randomisation & \hskip 0.12in0.057 (0.42) & \hskip 0.12in\textcolor{red}{0.00} & / \\  
\midrule
\multirow{ 3}{*}{ \textit{Population} } & age max & \hskip 0.08in-0.0032 (0.97) & \hskip 0.12in0.00 & / \\
& age min & \hskip 0.08in-0.044 (0.59) & \hskip 0.12in0.00 & / \\
& prise de médicaments & \hskip 0.12in0.025 (0.75) & \hskip 0.12in0.00 & / \\
\midrule
\multirow{ 9}{*}{ \textit{ \shortstack{Implementation \\ du \gls{nfb}} } } & nombre de sessions & \hskip 0.08in-0.14 (0.11) & \hskip 0.12in0.00 & \textcolor{red}{/} \\
& durée de la session & \hskip 0.08in-0.17 (0.12) & \hskip 0.12in0.00 & \textcolor{green}{\textbf{2$^{eme}$ noeud}} \\
& durée  du traitement & \hskip 0.12in\textbf{0.17 (0.054)} & \hskip 0.12in\textcolor{red}{0.00} & \textbf{2$^{eme}$ et 3$^{eme}$ noeuds} \\
& fréquence des sessions & \hskip 0.08in-0.081 (0.20) & \hskip 0.12in\textcolor{red}{0.00} & \textbf{2$^{eme}$ noeud} \\ 
& \gls{smr} & \hskip 0.12in0.070 (0.32) & \hskip 0.12in0.00 & / \\
& augmentation de beta en central & \hskip 0.12in0.18 (0.11) & \hskip 0.12in0.00 & / \\ 
& diminution de theta & \hskip 0.08in-0.13 (0.27) & \hskip 0.12in0.00 & / \\ 
& \gls{scp} & \hskip 0.08in-0.084 (0.60) & \hskip 0.12in0.00 & / \\ 
& phase de transfert & \hskip 0.12in\textbf{0.22 (0.036)} & \hskip 0.12in0.00 & \textcolor{green}{\textbf{1$^{er}$ noeud} }\\
\midrule
\multirow{ 2}{*}{ \textit{ \shortstack{Qualité de \\ l'acquisition} } } & plus d'une électrode active & \hskip 0.08in\textbf{-0.17 (0.018)} & \hskip 0.12in\textcolor{red}{0.00} & / \\ 
& \gls{eeg} qualité 2 & \hskip 0.08in\textbf{-0.26 (0.002)} & \hskip 0.12in\textcolor{red}{0.00} & \textcolor{red}{/} \\  
\midrule
\multirow{ 2}{*}{ \textit{Qualité du signal} } & \gls{eog} rejet ou correction & \hskip 0.08in\textcolor{red}{-0.12 (0.12)} & \hskip 0.12in0.00 & \textcolor{green}{\textbf{1$^{er}$ noeud}} \\
& Rejet des artefacts basé sur l'amplitude & \hskip 0.12in0.13 (0.12) & \hskip 0.12in0.00 & / \\   
\bottomrule
\end{tabular}
\end{center}

  \label{Table:table_factors_analysis_results_summary_update}
\end{table}

Les résultats obtenus sont globalement cohérents avec ceux présentés dans \citet{Bussalb2019a}. Alors que la \gls{wls} identifie les mêmes facteurs à l'exception de la correction ou rejet 
des artefacts oculaires (deuxième colonne de la Table~\ref{Table:table_factors_analysis_results_summary_update}), le \gls{lasso} (troisième colonne de la 
Table~\ref{Table:table_factors_analysis_results_summary_update}) est beaucoup plus sévère : tous les coefficients des facteurs sont mis à 0 sauf celui de \gls{pblind}. Ce facteur est ainsi 
le seul à être encore identifié par les trois méthodes qui s'accordent une nouvelle fois sur l'influence négative des évaluateurs aveugles sur les résultats.  

En ce qui concerne les deux autres facteurs précédemment identifiés par les trois méthodes : 
\begin{itemize}
\item la durée du traitement n'est plus retournée que par deux méthodes, 
\item la qualité de l'\gls{eeg} égale à 2 n'est plus retournée que par la \gls{wls}. 
\end{itemize}

Un seul autre facteur est identifié par deux méthodes : la présence d'une phase de transfert qui semblerait, selon la \gls{wls} et le \gls{dt}, avoir un impact négatif sur l'efficacité 
du \gls{nfb}. Ce facteur a été identifié dans \citet{Bussalb2019a} seulement par la \gls{wls} avec cette même direction d'effet.

\subsubsection{Discussion sur la mise à jour}

Entre le 12 février 2019 et le 2 septembre 2019, 8 nouvelles études répondant au critère d'inclusion de la \gls{saob} ont été publiées, ce qui montre l'importance de mettre à jour ce 
genre d'analyse. L'ajout de ces 8 études a modifié certaines conclusions de \citet{Bussalb2019a}. 

En effet, tout d'abord, la qualité de l'\gls{eeg} égale à 2 n'est plus retournée que par une seule méthode,
remettant en question son influence positive sur l'efficacité du \gls{nfb}. Toutefois, bien que ce facteur soit intéressant, sa définition (voir \ref{choix_des_facteurs}) manque de précision 
du fait du peu d'informations disponibles sur le matériel utilisé par les études. Ainsi, afin d'évaluer au mieux son influence, les études devraient fournir davantage d'informations. Ensuite, 
la durée du traitement n'est plus identifiée que par deux méthodes et la fréquence des sessions par une seule : on peut encore avancer, bien qu'avec moins de certitude que précédemment, 
qu'un traitement intensif est préférable. 

Alors que cette mise à jour a causé la perte de significativité de plusieurs facteurs, un facteur se voit identifié par deux méthodes : la présence d'une phase de transfert. 
Cette phase a pour but d'aider à transposer le contrôle appris lors des séances de \gls{nfb} à la vie de tous les jours, on pourrait donc s'attendre à un effet positif sur les symptômes
du \gls{tdah} comme le soulignent \citet{Arns2014, Strehl2006, Gani2008}, or la \gls{saob} conclut à un impact négatif. Ce résultat s'expliquerait peut-être par le fait qu'une phase de 
transfert seule ne soit pas si efficace si elle n'est pas couplée à l'utilisation d'une carte représentant la métaphore utilisée lors des sessions de \gls{nfb} qui permet à l'enfant de se 
souvenir plus facilement du contrôle qu'il exerçait durant la séance \citep{Bioulac2019, Bluschke2016}. Ce facteur devait être analysé mais étant trop homogène (plus de 80\% des observations n'utilisent pas de carte), il 
a été exclus de l'analyse. 

%Bien que la significativité avait été atteinte pour la qualité de l'\gls{eeg} égale à 2 et pour la durée du traitement par les trois méthodes, 
%les résultats ne sont pas figés : ils vont fluctuer jusqu'à se stabiliser quand un grand nombre d'observations sera disponible. Afin de mettre en évidence cette fluctuation, l'évolution 
%des p-values obtenues par la \gls{wls} en fonction des études incluses au fur et à mesure selon leur année de publication est représentée à la Figure~\ref{Figure:factors_evolution_pvalue} 
%pour la qualité de l'\gls{eeg} égale à 2 et pour la fréquence des sessions.
%
%\begin{figure}[h!]
  %\centering
	%\includegraphics[width=0.7\linewidth]{figures/chapter-3/factors-pvalue-evolution} 
  %\caption{Evolution des p-values obtenues par la régression linéaire pondérée \gls{wls} pour les coefficients associés à la fréquence des sessions (en vert) et 
	%à la qualité de l'\gls{eeg} égale à 2 (en bleu). Le seuil de significativité à 0.05 est représenté en pointillés rouges.}
  %\label{Figure:factors_evolution_pvalue}
%\end{figure}
%
%La p-value du coefficient associé à la qualité de l'\gls{eeg} égale à 2 se stabilise dès l'année 2014, alors que celle du coefficient associé à la fréquence des sessions 
%n'est plus stable à partir de l'année 2017. En effet, en plus de n'être plus significative, la p-value augmente énormément pour atteindre un pic à 0.30 et redescendre à environ 0.20
%avec l'ajout à la \gls{saob} des études publiées en 2019. Cette évolution montre bien que les résultats ne sont pas figés et donc pour que la \gls{saob} puisse apporter des indications
%fiables pour aider au design d'une étude évaluant l'efficacité du \gls{nfb} appliqué aux enfants \gls{tdah}, elle doit être régulièrement mise à jour. 

%% -> faire ce graphique pour la méta-analyse, faire une ref ici à ce graphique pour dire que dans le cas de la saob aussi la p-avalue fluctue et qu'il faut donc continuer
%% de la mettre à jour pour obtenir des résultats fiables

\section{Conclusion}

La \gls{saob} a permis d'identifier un facteur lié au matériel d'acquisition de l'\gls{eeg} comme ayant une influence positive sur l'efficacité du \gls{nfb}, 
ce qui indique fortement qu'elle est liée à un mécanisme d'action basé sur la modulaion de l'\gls{eeg}. Toutefois, la mise à jour de la \gls{saob} a rendu
plus incertain l'impact de ce paramètre dont la définition souffre du peu d'informations fourni par les études à ce sujet. 

L'intensité du traitement a aussi été trouvée comme contribuant à l'efficacité du \gls{nfb}, ce qui va dans le sens de ce qui est connu à propos de 
la théorie de l'apprentissage \citep{Mowrer1960} : un entrainement plus intense mène à une efficacité clinique augmentée. 

Alors que ces résultats contribuent certainement au débat, ce travail montre également que l'ultime démonstration de la preuve de l'efficacité du \gls{nfb}
appliqué aux enfants \gls{tdah} n'est pas encore atteinte, étant donné que les évaluations des enseignants ont été en partie invalidées comme indicatrices de 
l'effet placebo. Par conséquent, se référer aux résultats des évaluateurs \gls{pblind} pour mettre en évidence la spécificité de l'efficacité clinique n'est pas 
recommandé : il serait préférable d'avoir recours au sham-\gls{nfb} et à l'analyse des changements de l'\gls{eeg} en fonction des neuromarqueurs entrainés.

Comme l'a montré la mise à jour des résultats de la \gls{saob}, les conclusions de cette analyse peuvent encore évoluer. Au vu du nombre important 
d'études publiées sur le \gls{nfb} appliqué aux enfants \gls{tdah}, certains facteurs vont pouvoir finir par être inclus dans la \gls{saob}, comme par exemple 
l'individualisation du protocole de \gls{nfb} qui semble prometteuse. Le chapitre suivant va justement s'intéresser à la distibution d'un marqueur de l'attention, 
le \gls{tbr}, au sein d'une large population afin de déterminer si une personnalisation basée sur ce neuromarqueur est envisageable.




 