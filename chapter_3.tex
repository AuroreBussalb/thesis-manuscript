\chapter{Identification des facteurs influençant l'efficacité du Neurofeedback}

\section*{Introduction}

La réplication et la mise à jour de la méta-analyse de \citet{Cortese2016} décrite dans le chapitre précédent a permis de mettre en évidence la forte hétérogénéité des études incluses dans ce type d'analyse. 
En effet, même si ces études satisfont toutes le critère d'inclusion défini par les auteurs, elles diffèrent d'un point de vue technique et méthodologique : elles ont été rassemblées 
sans tenir compte par exemple de la qualité de l'acquisition de l'\gls{eeg}, du neuromarqueur entrainé lors du \gls{nfb} et du design de l'étude clinique (notamment le nombre de 
sessions et la durée du traitement). 

Afin de pallier ces limitations, une nouvelle approche a été implémentée : la \gls{saob} qui va justement tirer avantage de cette hétérogénéité. L'efficacité du traitement, quantifiée 
par l'\gls{es}-intra-groupe, de chaque intervention est considérée comme variable dépendante expliquée par des variables indépendantes qui sont ici les facteurs méthologiques et techniques. 
Le but de cette analyse est de déterminer les facteurs qui ont une influence sur l'efficacité du \gls{nfb} : au vu des résultats des précédentes méta-analyses \citep{Micoulaud2014, Cortese2016}, 
on peut notamment s'attendre à ce que les évaluateurs aveugles soient associés à une plus faible efficacité du traitement. 

\section{Extraction et pré-traitement des facteurs}

La première étape de la \gls{saob} est d'extraire les facteurs des études sélectionnées. Une liste de facteurs ayant potentiellement une influence sur l'efficacité du \gls{nfb} a été 
établie, puis les facteurs ont été extraits de chaque étude. Avant de débuter l'analyse, les facteurs sont pré-traités en suivant les étapes décrites dans cette section. 

\subsection{Choix des facteurs}

Les facteurs ayant une possible influence sur l'efficacité du \gls{nfb} ont été répartis en cinq catégories :
\renewcommand{\labelitemi}{$\bullet$}
\begin{itemize}
\item \emph{les biais méthodologiques :} la présence d'un groupe contrôle, l'aveugle des évaluateurs, la randomisation des sujets dans les essais contrôlés, et la validation de l'étude 
par un \gls{irb},
\item \emph{la population :} prise de psychostimulants durant le traitement par \gls{nfb}, la tranche d'âge des enfants inclus, la sévérité des symptômes du TDAH à pré-test (score clinique à pré-test
divisé par le score maximal à atteindre sur l'échelle clinique),
\item \emph{l'implémentation du \gls{nfb} :} le protocole utilisé (\gls{scp}, \gls{smr}, l'augmentation du rythme theta, l'augmentation du rythme beta dans les aires centrales ou frontales 
et la diminution du rythme theta), la présence d'une phase de transfert lors de l'entrainenement par \gls{nfb}, l'utilisation d'une carte de transfert pour s'entrainer à la maison ou à l'école, 
le type de seuillage pour les récompenses discrètes, le nombre de sessions de \gls{nfb}, la durée et la fréquence des sessions, la durée du traitement, l'individualisation des bandes de fréquence
basée sur l'\gls{iapf}, et le couplage du \gls{nfb} avec l'\gls{emg}-Biofeedback.
\item \emph{la qualité de l'acquisition :} la présence d'une ou plus électrode active et la qualité de l'\gls{eeg}. Cette dernière est un indicateur allant de 1 à 3, calculé sur les critères 
suivants : 
\begin{description} 
\item[le type d'électrode utilisée :] \gls{agcl}/Gel ou \gls{au}/Gel,
\item[le contrôle de l'impédance :] la vérification du bon contact entre la peau et les électrodes en gardant l'impédance inférieure à $40$k$\Omega$,
\item[la certification du matériel hardware utilisé :] le matériel doit être conforme à la norme ISO-60601-2-26 \citep{ISO}.
\end{description}

Un score de qualité de 3 est donné si tous les critères ci-dessus sont remplis. Si au moins l'un d'eux est satisfait, le score est de 2, sinon il sera mis à 1.

\item \emph{la qualité du signal} : rejet en temps réel (l'\textit{epoch} est rejeté, pas de retour calculé) ou correction (retour calculé sur l'\textit{epoch} débruité) des 
artefacts oculaires et rejet en temps réel d'artefacts génériques détectés grâce à leur large amplitude. 
\end{itemize}

Afin d'éviter tout biais, le nom des facteurs a été caché durant les analyses : il n'a été révélé que lorsque le modèle a été considéré comme valide notamment au niveau 
de la normalisation des variables et de la validation des hypothèses du modèle.  

\subsection{Pré-traitement des facteurs}

Les auteurs des études incluses dans la \gls{saob} ne précisent pas systématiquement toutes les valeurs des facteurs, ce qui conduit à des observations manquantes. Afin que 
les facteurs pour lesquels peu d'observations sont diponibles ne faussent pas l'analyse, un critère d'exclusion arbitraire a été mis en place : si pour un facteur le nombre d'observations 
manquantes excède plus de 20\% du nombre total d'observations, ce facteur est exclu. 

Par ailleurs, comme cette analyse tire avantage de l'hétérogénéité des études, si un facteur a plus de 80\% d'observations identiques, celui-ci est également exclu. 

Il est important de noter qu'une étude ne correspond pas nécessairement à une observation : lorsque plusieurs échelles cliniques et/ou évaluateurs sont disponibles dans une étude,
chaque couple échelle clinique-évaluateur est considéré comme une observation.

Les facteurs qui sont des variables catégorielles (le protocole utilisé par exemple) sont codés en \textit{dummies} : la présence du facteur est représentée par un 1 et son absence par 0. 

Enfin, toutes les variables sont standardisées : à chaque observation est soustraite la moyenne de l'ensemble des observations, le tout divisé par l'écart-type de la moyenne de 
l'ensemble des observations. Cette étape n'est pas appliquée dans le cadre de l'arbre de décision.

Les facteurs sont les variables indépendantes de l'analyse et la variable dépendante quantifiant l'efficacité du \gls{nfb} est décrite dans la section suivante.


\section{Explication de l'efficacité du Neurofeedback par des méthodes multivariées}

\subsection{Calcul de la taille d'effet intra-groupe}

L'\gls{es}-intra-groupe est calculée à partir des moyennes et écart-types des scores cliniques totaux donnés par les parents et les enseignants. De plus, lorsqu'une étude 
donne des résultats pour plus d'une échelle clinique, l'\gls{es}-intra-groupe est calculé pour chaque échelle :
\begin{equation*}
\label{eq:factors_effect_size_within_subject}
\text{ES} = \frac{M_{\text{post,T}} - M_{\text{pre,T}}}{\sqrt{\frac{\sigma_{\text{pre,T}}^2 + \sigma_{\text{post,T}}^2}{2}}},
\end{equation*} 
\noindent où $M_{\text{t,T}}$ est la moyenne sur l'échelle clinique, pour le traitement T, au moment t (pré-test ou post-test) et $\sigma_{\text{t,T}}$ représente
son écart-type. Avec cette défintion de l'\gls{es}, on se concentre sur l'effet du traitement au sein du groupe \citep{Cohen1988} précédemment utilisé dans la littérature sur le \gls{nfb} 
\citep{Arns2009, Maurizio2014, Strehl2017}.  
%Finally, to avoid to break analysis methods assumptions, an outlier rejection was applied defining thresholds of acceptance as 
%$[\mu - 3 \sigma, \mu + 3 \sigma]$, with $\mu$ and $\sigma$ being respectively the mean and the standard deviation of all within-\gls{es} computed \citep{Shewhart1931}.
%
%The within-\gls{es} was then considered as a dependent variable to be explained by the factors (the independent variables). 
%The following three methods, implemented with the Scikit-Learn Python \citep[version 0.18.1]{Pedregosa2011} and the Statsmodels Python
%\citep[version 0.8.0]{Seabold2010} libraries, were used to perform the regression:
%\begin{itemize}
  %\item weighted multiple linear regression with \gls{wls} \citep{Montgomery2012};
	%\item sparsity-regularized linear regression with \gls{lasso} \citep{Tibshirani1996};
	%\item decision tree regression \citep{Quinlan1986}.
%\end{itemize}

\subsection{Pré-traitement des facteurs}
