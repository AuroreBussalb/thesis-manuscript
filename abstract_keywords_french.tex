% Résumé en français, 4000 mots max 
% mots clés

\begin{center}
\MakeUppercase{\LARGE{R}\Large{esume de these}} \\
\vspace{0mm}
\noindent\rule{16cm}{0.4pt}
\end{center}

% une phrase d'accroche
Cette thèse porte sur le \glsfirst{nfb}-\glsfirst{eegi} appliqué au \gls{tdah} chez l'enfant dont l'efficacité et les 
facteurs d'amélioration sont étudiés en se basant sur des données cliniques et sur les signaux \gls{eeg}. 

% définitions
Le \gls{nfb} est une technique d'apprentissage à visée thérapeutique permettant de modifier l'activité \gls{eeg}, 
dans le cadre d'un paradigme de conditionnement présentant des récompenses auditives et/ou visuelles via une 
interface de jeu. Cette technique a un large champ d'applications. Ce travail s'intéresse spécifiquement à son utilisation
thérapeutique dans le cadre 
du \gls{tdah} chez l'enfant. Ce trouble neuro-développemental se caractérise par un déficit attentionnel et/ou une hyperactivité 
motrice qui impactent négativement le bien-être des enfants. Le traitement médicamenteux est communément prescrit du fait 
de son efficacité mais peut entrainer des effets secondaires, ce qui mène certains parents et pédopsychiatres à se tourner 
vers des approches non-médicamenteuses telles que le \gls{nfb}.
 
% problématique
Le \gls{nfb}-\gls{eegi} appliqué au traitement du \gls{tdah} chez l'enfant a fait l'objet de nombreuses études ayant pour but de démontrer son efficacité. 
Cependant, elles n'ont pas encore permis d'atteindre de réel consensus, ce qui pourrait s'expliquer par le nombre restreint de patients inclus 
mais aussi par la divergence clinique, technique et méthodologique des études. 

% méthodes
Ainsi l'objectif de cette thèse est, tout d'abord, d'effectuer un état des lieux du niveau de preuves de l'efficacité 
du \gls{nfb} pour les enfants \gls{tdah} grâce à la réplication et à la mise à jour d'une méta-analyse sur le sujet. 
La réplication a pour but de trancher certains débats autour des choix méthodologiques de l'auteur. 
La mise à jour de la méta-analyse permet l'inclusion de davantage d'études qui, malgré l'hypothèse concernant leur homogénéité,
diffèrent les unes des autres d'un point de vue des populations et de la méthode.  

Cette hétérogénéité est alors étudiée afin d'identifier des facteurs méthodologiques, cliniques et techniques qui influenceraient 
la performance de ce traitement. Cette étape 
est réalisée à l'aide de trois méthodes de régression multivariées qui associent l'efficacité du \gls{nfb} aux facteurs d'intérêt. 
Les résultats de chacune de ces méthodes sont combinés afin de déterminer quels paramètres parmi ceux étudiés auraient effectivement un impact. 

Cependant, certains facteurs innovants ne peuvent être étudiés dans cette analyse car trop peu d'études en rapportent la performance. 
Ainsi, la dernière étape de ce travail consiste à analyser la pertinence de la personnalisation du traitement par \gls{nfb} selon le profil 
\gls{eeg} des enfants à l'aide de méthodes de partitionnements appliquées à un marqueur \gls{eeg} couramment associé au \gls{tdah}. 

% résultats
La réplication et la mise à jour de la méta-analyse confirment les résultats obtenus par le passé sur davantage d'études : 
les évaluateurs non aveugles au traitement suivi par l'enfant notent une diminution significative des symptômes, à l'inverse 
de ceux qui sont aveugles pour des raisons peut-être plus complexes qu'à première vue. Ce résultat est confirmé par l'analyse 
des facteurs pouvant avoir une influence sur l'efficacité du \gls{nfb}, qui identifie également l'intensité du traitement. Enfin, 
personnaliser le traitement par \gls{nfb} semble pertinent étant donné que l'analyse de partitionnement distingue deux groupes d'enfants 
\gls{tdah} avec des phénotypes \gls{eeg} distincts.

% conclusion
Pour conclure, ce travail contribue à apporter des réponses sur l'efficacité du \gls{nfb} en répliquant et en mettant à jour une méta-analyse 
ainsi qu'en se penchant sur les évaluations des personnes dites aveugles au traitement. Par ailleurs, ce travail a permis de déterminer des paramètres dont la définition ou la présence 
pourrait améliorer l'efficacité du \gls{nfb}.

\large{\textbf{Mots-clés}} : \glsfirst{nfb}, \glsfirst{eeg}, \glsfirst{tdah}, meta-analyse, analyse de biais, partitionnement.