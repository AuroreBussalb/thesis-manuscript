% Résumé en français, 4000 mots max 
% mots clés

\begin{center}
\MakeUppercase{\LARGE{R}\Large{esume de these}} \\
\vspace{0mm}
\noindent\rule{16cm}{0.4pt}
\end{center}


% une phrase d'accroche
Cette thèse porte sur le \glsfirst{nfb}-\glsfirst{eegi} appliqué au \glsfirst{tdah} chez l'enfant dont l'efficacité et les facteurs d'amélioration sont 
étudiés ici en se basant d'une part sur les données cliniques et d'autre part sur les signaux électroencéphalographiques. 
% définitions
Le \gls{nfb} est une technique d'apprentissage à visée thérapeutique permettant 
de modifier un paramètre d'activité cérébrale, ici l'\gls{eeg}, au moyen d’un système de récompenses auditives et/ou visuelles 
délivrées instantanément via une interface de jeu sérieux. Cette technique a un large champ d'applications, cependant ici elle est uniquement étudiée 
dans le cadre du \gls{tdah} chez l'enfant. Le \gls{tdah} est un trouble neuro-développemental chronique communément traité par la prise de psychostimulants, 
mais dont le risque d'effets secondaires mène certains parents et pédopsychiatres à se tourner vers des approches non-médicamenteuses telles que le \gls{nfb}. 
% problématique
Le \gls{nfb} appliqué au traitement du \gls{tdah} chez l'enfant a fait l’objet de nombreuses études ayant pour but de démontrer son efficacité, sans avoir encore atteint de réel
consensus, ce qui pourrait s'expliquer par l'intensité des divergences méthodologiques, cliniques et techniques de ces études. 
% méthodes
Ainsi l'objectif de cette thèse est, dans un premier temps, d'effectuer l'état des lieux du niveau de preuves de l'efficacité du \gls{nfb} pour les enfants \gls{tdah} grâce
à la réplication et à la mise à jour d'une récente méta-analyse sur le sujet à l'aide d'un package Python. Dans un 
deuxième temps, l'hétérogénéité des études dans le domaine d'intérêt est tournée à notre avantage afin de permettre 
l'identification des facteurs méthodologiques, cliniques et techniques qui pourraient avoir une influence sur la performance de ce traitement. Cette étape est réalisée
à l'aide de trois méthodes multivariées qui associent l'efficacité du \gls{nfb}, quantifiée par une taille d’effet intra-sujet, aux facteurs d'intérêt. 
Les résultats de chacune de ces méthodes sont combinés afin de déterminer quels paramètres parmi ceux étudiés auraient effectivement un impact. 
Enfin, la pertinence de personnaliser le traitement par \gls{nfb} selon le profil \gls{eeg} des enfants, paramètre qui pourrait améliorer les résultats de ce traitement, est étudiée dans un troisième
temps à l'aide de méthodes de partitionnements appliquées à un marqueur \gls{eeg} extrait d'une large population d'enfants \gls{tdah} couramment associé à ce trouble. 
% résultats
La réplication et la mise à jour de la méta-analyse confirment les résultats obtenus par le passé sur davantage d'études : les évaluateurs non aveugles au traitement suivi par 
l'enfant notent une diminution significative des symptômes à l'inverse de ceux qui sont aveugles. L'analyse des facteurs pouvant avoir une influence sur l'efficacité du \gls{nfb}
est en accord avec ce résultat et par ailleurs conclut qu'un traitement intensif est préférable et qu'inclure une période où aucun retour n'est délivré à l'enfant durant ses session de \gls{nfb},
diminuerait l'efficacité du traitement. Enfin, personnaliser le traitement par \gls{nfb} semble pertinent étant donné que notre analyse de partitionnement a identifié deux groupes
d'enfants \gls{tdah} selon leur profil \gls{eeg}.
% conclusion
Grâce à ces travaux, des facteurs d'amélioration du traitement par \gls{nfb} des enfants \gls{tdah} ont été mis en évidence. 



\large{\textbf{Mots-clés}} : \glsfirst{nfb}, \glsfirst{eeg}, \glsfirst{tdah}, meta-analyse, analyse de biais, partitionnement.