\chapter*{Remerciements}

Toute thèse de doctorat implique son lot d'embûches mais aussi de succès et, dans un cas comme dans l'autre, des personnes ont toujours 
été présentes, je tiens donc à les en remercier. 

Je tiens d'emblée à exprimer ma gratitude envers les membres du jury de thèse pour le temps qu'ils ont consacré à la lecture de mon manuscrit et pour leurs 
remarques.

Ensuite, mes remerciements vont à mon directeur de thèse, le Pr. Richard Delorme, pour son travail de supervision tout au long
de ma thèse. Je remercie également le Dr. Eric Acquaviva pour sa disponibilité et pour l'intérêt qu'il a porté à mon travail. 

Ce travail n'aurait pas pu être mené sans l'accueil chaleureux, la bienveillance et le soutien de l'équipe Recherche de Mensia Technologies, notamment  
du Dr. Louis Mayaud, mon superviseur scientifique en entreprise, que je remercie pour son encadrement émulateur et 
pour avoir continué de s'intéresser à mon travail et de m'encourager malgré ses nombreuses
responsabilités. Je tiens également à remercier tout particulièrement le Dr. Quentin Barthélemy pour ses conseils 
avisés et ses idées judicieuses pour améliorer mon travail et approfondir mes analyses, ainsi que pour le temps qu'il m'a consacré. 

Plus généralement, je remercie l'ensemble de mes collègues de Mensia Technologies avec qui cela a été un plaisir de travailler et dont la bonne 
humeur a permis d'adoucir un contexte parfois délicat.

Je tiens également à exprimer ma gratitude envers le comité scientifique de Newrofeed et le Dr. Marco Congedo pour leurs remarques constructives sur mon travail. 

Ensuite, je souhaite aussi remercier l'équipe de MyBrain Technologies pour leur gentillesse, en particulier le Dr. Yohan Attal et le Dr. Xavier Navarro-Sune 
pour m'avoir accueillie dans leurs bureaux. 

Enfin, je remercie mes amis et ma famille pour leurs encouragements sans cesse renouvellés et leur soutien. Pour finir, je tiens à remercier 
mon compagnon, Anthony Jabbour, dont la présence indéfectible à mes côtés m'a permis de tenir jusqu'au bout.     