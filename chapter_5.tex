\chapter{Conclusion et perspectives} \label{chapitre-5}

\section{Résumé des contributions de la thèse}

Les différentes analyses effectuées au cours de cette thèse ont contribué à :
\begin{itemize}
\item trancher le débat entourant les choix effectués dans une récente méta-analyse grâce à une analyse de sensibilité \citep{Cortese2016, Micoulaud2016} (cf. chapitre \ref{chapitre-2}),
\item étudier l'efficacité du \gls{nfb} en mettant à jour cette même méta-analyse pour y inclure les nouvelles études publiées \citep{Cortese2016} (cf. chapitre \ref{chapitre-2}),
\item déterminer les paramètres inhérents à la mise en place du traitement par \gls{nfb} qui influencent sa performance (cf. chapitre \ref{ch-saob}),
\item mettre en évidence la différence entre les évaluations des personnes aveugles et non-aveugles (cf. chapitre \ref{ch-saob}),
\item proposer un package Python accessible en ligne permettant d'effectuer des méta-analyses et de mettre à jour l'analyse des facteurs \citep{Bussalb2019c} (cf. chapitres \ref{chapitre-2} 
et \ref{ch-saob}),
\item dresser un état des lieux des définitions du \gls{tbr} (cf. chapitre \ref{chapitre-4}),
\item analyser la distribution du \gls{tbr} au sein d'une population d'enfants \gls{tdah} pour évaluer la pertinence 
de la personnalisation du \gls{nfb} (cf. chapitre \ref{chapitre-4}).
\end{itemize}

\section{Conclusion}

Ce travail a permis de contribuer au domaine du \gls{nfb} appliqué aux enfants souffrant de \gls{tdah} en menant trois analyses 
distinctes dont les principaux résultats sont résumés ici. 

Tout d'abord, la méta-analyse la plus récente au moment où ce travail a été réalisé, celle de \citet{Cortese2016}, a été répliquée 
pour quantifier l'impact des choix des auteurs qui ont été débattus par la communauté scientifique \citep{Micoulaud2016}. Par la suite, cette méta-analyse est 
mise à jour. Ces deux étapes ont été effectuées à l'aide d'un package Python développé 
dans le cadre de cette thèse. Il se trouve que les choix effectués par \citet{Cortese2016} n'ont pas d'impact sur les conclusions, 
mais illustrent la complexité de l'émergence de la preuve clinique.

Les résultats de la mise à jour confirment ceux obtenus par \citet{Cortese2016} : 
l'efficacité perçue est d'autant plus faible qu'elle est évaluée par 
des observateurs distants de l'intervention et de son ou sa bénéficiaire, comme les enseignants. Alors que ce phénomène très documenté est souvent 
interprété comme un reflet des biais positifs dont sont victimes les parents du fait de leur plus grande implication dans la délivrance de 
l'intervention, une analyse approfondie révèle que les enseignants sont également moins sensibles aux symptômes en général et en particulier 
à leur évolution lente \citep{Sollie2013, Narad2015}. 
Cela remet profondément en cause les présupposés méthodologiques actuels de l'évolution des interventions non-médicamenteuses. Par ailleurs
lorsqu'on étudie précisément l'évolution de l'\glsfirst{est} et de sa $p$-value au fur et à mesure de l'inclusion des études selon leur année de publication, 
on remarque que ces valeurs ne se sont pas stabilisées (visuellement pour l'\gls{est} et pour la $p$-value grâce à son intervalle de confiance), 
ce qui appelle à de nouvelles mises à jour.

Ce travail a été l'occasion d'explorer 
la littérature sur le \gls{nfb} et ainsi de souligner les divergences des choix cliniques, méthodologiques et techniques effectués dans les études 
d'efficacité qui impactent la fiabilité
des résultats obtenus dans les méta-analyses. C'est pourquoi une approche tirant avantage de cette hétérogénéité a été mise en place : la \glsfirst{saob}.

La \gls{saob} a recours à des méthodes multivariées pour déterminer les facteurs cliniques, méthodologiques et/ou techniques qui pourraient avoir une 
influence sur l'efficacité du \gls{nfb}. 

Cette analyse permet de conclure qu'un traitement intensif (c'est à dire court mais avec un rythme important de sessions par semaine) augmenterait 
l'efficacité du \gls{nfb}. Ce résultat est aussi bien en accord avec un mode d'action spécifique du \gls{nfb} qu'avec la présence d'un effet placebo :
on peut s'attendre que l'amplitude de cet effet augmente avec l'implication des enfants. A l'inverse,  
intégrer une phase de transfert durant la session diminuerait son efficacité ce qui est contre intuitif et préoccupant : on s'attendrait à ce que ce facteur 
soit neutre ou positif, cet effet négatif peut remettre en cause la qualité des données utilisées ou la méthode. Toutefois, les autres résultats obtenus 
avec la \gls{saob} étant cohérents, le cas de l'intégration d'une phase de transfert est peut-être un faux positif. En effet, comme attendu, les évaluations des enseignants ne sont pas en faveur de l'efficacité du \gls{nfb}. 
Cette différence entre les deux types d'évaluateurs peut, à première vue, s'expliquer par l'effet placebo mais une analyse plus approfondie des 
résultats de la \gls{saob} montre que cette explication n'est pas totalement valable, ce qui est en accord avec les conclusions de la partie précédente. 
Ici aussi, la mise à jour de ces résultats est nécessaire : plus la \gls{saob} sera appliquée sur un nombre important d'observations,
plus ses résultats seront fiables.

La \gls{saob} a permis d'étudier l'impact de différents facteurs sur la performance du \gls{nfb}, cependant la personnalisation 
des protocoles d'entrainement dont l'influence serait intéressante 
à explorer, a été exclue de cette analyse, faute d'un nombre suffisant d'études la proposant. Ainsi, la pertinence de la personnalisation 
du \gls{nfb} a été étudiée ici à l'aide de l'analyse de la distribution 
du \gls{tbr} dans une large population d'enfants \gls{tdah}. La variabilité des définitions du \gls{tbr} est tout d'abord mise en 
évidence, puis une recommandation quant à son calcul est proposée. A partir de cette définition, les \gls{tbr} de la population à notre diposition
sont obtenus puis partitionnés à l'aide de trois méthodes différentes. Ces dernières se sont accordées sur le fait que cette distribution 
est bimodale, ce qui va dans 
le sens d'une personnalisation de protocoles basée sur la valeur de \gls{tbr} dont la valeur seuil conduisant au meilleur équilibre entre 
\glsfirst{fpr} et \glsfirst{tpr} est de 4.1. 

Grâce à ce résultat, la présence de 
groupes phénotypiques \gls{eegi} dans la population d'enfants souffrant de \gls{tdah} est confirmée, sans nécessairement prouver la valeur ajoutée d'une 
intervention personnalisée. Cela constitue néanmoins un premier pas vers l'identification de ces populations pour en 
permettre la stratification dans le cadre d'études cliniques (NEWROFEED, NCT02778360, Mensia Technologies, France, ClinicalTrials.gov, \citet{Bioulac2019} et NCT02251743, Arnold, Ohio State University, 
ClinicalTrials.gov, \citet{Kerson2013}).

Ainsi, le travail présenté dans ce manuscrit a permis de décrire avec précision comment l'efficacité du \gls{nfb} est évaluée à l'aide de 
la technique de la méta-analyse et de donner des 
directions pour choisir les paramètres permettant d'obtenir une meilleure performance du \gls{nfb}. Par ailleurs, les packages Python développés pour mener 
la méta-analyse et la \gls{saob}
sont accessibles en ligne et permettent à tout chercheur de relancer ou mettre à jour l'analyse : il suffit simplement d'ajouter de nouvelles 
entrées dans le fichier \gls{csv}.

A l'issue des analyses menées ici, l'efficacité du \gls{nfb} dans le cadre du \gls{tdah} chez l'enfant n'est toujours pas clairement démontrée. 
Le fait qu'on ne puisse pas encore conclure sur ce point interpelle : malgré des années de recherches de plus en plus rigoureuses, aucune conclusion
clairement en faveur ou en défaveur n'a été énoncée, ce qui montre la complexité du domaine.   

Les deux articles scientifiques publiés au cours de cette thèse ont contribué au domaine du \gls{nfb} appliqué aux enfants \gls{tdah}
comme le soulignent les citations dont ils ont fait l'objet. L'article traitant de la mise à jour de la méta-analyse de \citet{Cortese2016} 
\citep{Bussalb2019clinical} a été cité notamment dans le cadre des débats autour des évaluations des enseignants \citep{Bottinger2020} et pour ses résultats
de la mise à jour de la méta-analyse \citep{Bluschke2020}. L'article sur l'analyse de la distribution des \gls{tbr} des enfants souffrant du \gls{tdah} 
\citep{Bussalb2019clinical}
a été cité concernant ses recommandations quant au calcul du \gls{tbr} \citep{Bioulac2020}. 

\section{Perspectives}

Les résultats présentés ici, notamment ceux de la méta-analyse et de la \gls{saob}, ont vocation à être mis à jour afin d'inclure davantage 
d'études : les résultats alors obtenus se stabiliseront et leur fiabilité augmentera.   

Jusqu'à maintenant, les méta-analyses sont des outils issus des statistiques fréquentistes,
évaluant uniquement le degré de compatibilité entre les données et une hypothèse nulle.
Les courbes d'évolution des tailles d'effets et des $p$-values nous invitent à basculer vers des méta-analyses bayésiennes,
attribuant une probabilité \textit{a priori} aux hypothèses testées (en utilisant l'état de l'art) et permettant ainsi d'estimer 
la probabilité \textit{a posteriori} que l'hypothèse nulle soit vraie.
Une telle approche a déjà été décrite et implémentée \citep{Dormuth2016, Spiegelhalter2004} mais n'a pas encore été 
utilisée dans le cas du \gls{nfb} appliqué au \gls{tdah}.  

Par ailleurs, une mise à jour de la \gls{saob} pourra permettre d'étudier de nouveaux facteurs, apportant ainsi de nouvelles  
indications. Dans sa forme actuelle, la \gls{saob} donne des recommandations qualitatives sur comment choisir les facteurs, 
il serait peut-être intéressant de mettre en place des méthodes conduisant à des directives quantitatives.

Les conclusions de l'analyse de la distribution des valeurs de \gls{tbr} issues d'une large population d'enfants \gls{tdah} justifieraient le recours à la 
personnalisation des protocoles de \gls{nfb} selon le profil \gls{eegi} de l'enfant. Ce genre d'approche est de plus en plus utilisée, ainsi davantage 
d'analyses s'intéressant
à la pertinence et à l'efficacité d'une telle approche vont être menées : il serait notamment important de déterminer si l'âge des enfants 
a effectivement influencé l'étude
présentée dans ce manuscrit. Par ailleurs, appliquer ces analyses sur des valeurs de \gls{tbr} obtenues les yeux fermés serait à envisager. 

Dans ce manuscrit, le \gls{nfb} a été étudié pour les enfants souffrant du \gls{tdah} mais les analyses effectuées ici pourraient être 
transposées à une toute autre application comme celles listées en \ref{NFB_applications}.
En particulier, la \gls{saob} pourrait être adaptée de façon à inclure toutes les études sur l'efficacité du \gls{nfb} 
quel que soit le trouble à traiter, pour déterminer si des facteurs généraux sont identifiés comme suggéré en \ref{conclusion_saob}. 

Enfin, certaines questions majeures concernant le \gls{nfb} n'ont pas été traitées dans cette thèse,
à l'instar de la définition et de la mise en place d'un protocole \textit{sham}-\gls{nfb}, déployable au sein d'un essai clinique en double aveugle. 
Un tel protocole 
aurait pour but de contrôler pour un éventuel effet placebo, effet qui est au coeur du débat sur l'efficacité du \gls{nfb} appliqué aux enfants \gls{tdah}. 

La recherche dans le domaine du \gls{nfb} est très active, aussi bien au niveau de la compréhension des mécanismes neurophysiologiques 
qu'à celui de son efficacité clinique.
Cependant, malgré le nombre important d'études menées sur le sujet, peu sont nettement en faveur de son efficacité. 
Ce constat pourrait s'expliquer par le fait que, depuis tant d'années de recherche, la plupart des études comportent un nombre
restreint de sujets et ne sont parfois pas contrôlées : il aurait sans doute été préférable de réduire le nombre de petites études pour financer une grande étude
qui aurait inclus davantage de sujets. Cette limitation n'est pas spécifique au domaine du \gls{nfb} : elle a été notamment mise en évidence lors de la crise du 
Covid-19 \citep{Gautret2020, Sanders2020}.

Par ailleurs, il est possible qu'appliquer le \gls{nfb} au \gls{tdah} ne soit pas pertinent : ce trouble s'accompagne généralement de troubles de l'apprentissage,
or le \gls{nfb} demande à l'utilisateur d'apprendre à réguler son activité cérébrale. Ainsi, le \gls{tdah} ne serait peut-être pas l'application du 
\gls{nfb} à privilégier. En effet, le \gls{nfb} parait prometteur dans d'autres domaines, notamment la gestion la douleur \citep{Mayaud2019}, sur lesquels 
il faut continuer de se pencher.

Le moyen de trancher la question de l'efficacité du \gls{nfb} pour les enfants souffrant du \gls{tdah} serait de mettre en place des études randomisées 
et contrôlées incluant de grands échantillons pour augmenter la puissance statistique. De plus, l'environnement technique (c'est à dire 
les électrodes, la stabilité et la qualité du signal, l'extraction de neuromarqueurs pertinent...) doit être parfaitement maîtrisé pour espérer observer
l'efficacité du \gls{nfb}. Cependant, encore trop peu d'études remplissent cette condition ce qui ajoute de la difficulté pour répondre à cette question d'efficacité. 
Ainsi, le problème principal réside peut-être moins dans la capacité à mener des analyses de données que dans le fait de savoir si les données utilisées 
ont vraiment du sens.  

Les récentes études cliniques évaluant l'efficacité du \gls{nfb} 
font preuve de plus de rigueur : elles incluent de plus en plus de sujets et suivent une méthodologie davantage fiable \citep{Bioulac2019}. 
En effet, la communauté scientifique appelle à être le plus transparent possible, notamment sur les caractéristiques du \textit{feedback} afin d'aider à 
conclure quant à l'efficacité du traitement \citep{Ros2019}. Par conséquent, dans quelques années, il sera sans doute possible de déterminer si le \gls{nfb} 
appliqué aux enfants souffrant de \gls{tdah} est efficace ou non.



