\chapter{Conclusion et perspectives} \label{chapitre-5}

\section{Conclusion}

Cette thèse avait pour but d'étudier les facteurs de réussite de l'entrainement par \gls{nfb} appliqué aux enfants \gls{tdah} et, pour ce faire, trois objectifs ont été identifiés :
\begin{enumerate}
\item étudier l'efficacité du \gls{nfb} à l'aide d'une méta-analyse,
\item déterminer les paramètres inhérents à la mise en place du traitement par \gls{nfb} qui influencent sa performance,
\item analyser la distribution du \gls{tbr} au sein d'une population d'enfants \gls{tdah} pour évaluer la pertinence de la personnalisation du \gls{nfb}.
\end{enumerate}

Pour atteindre le premier objectif, la méta-analyse la plus récente au moment où ce travail a été réalisé, celle de \citep{Cortese2016}, a été répliquée et mise à jour à l'aide d'un package
Python développé dans le cadre de cette thèse. Les résultats de cette première étape confirment ceux obtenus par \citep{Cortese2016} : les parents, qui savent quel traitement suivent leurs enfants, 
observent une amélioration de leurs symptômes, ce qui n'est pas le cas des enseignants, qui eux sont considérés comme probablement aveugles au traitement. Toutefois, lorsqu'on étudie
précisément l'évolution de l'\glsfirst{est} et de sa $p$-value au fur et à mesure de l'inclusion des études selon leur année de publication, on remarque que ces valeurs ne se sont pas 
stabilisées, ce qui appelle à de nouvelles mises à jour. 

Ce travail a été l'occasion d'explorer 
la littérature sur le \gls{nfb} et ainsi de souligner l'hétérogénéité des choix cliniques, méthodologiques et techniques effectués dans les études d'efficacité qui impactent la fiabilité
des résultats obtenus dans les méta-analyses. C'est pourquoi une approche tirant avantage de cette hétérogénéité a été mise en place : la \gls{saob}.

La \gls{saob} a recours à des méthodes multivariées pour déterminer les facteurs cliniques, méthodologiques et/ou techniques qui pourraient avoir une influence sur l'efficacité du \gls{nfb}. 
Cette analyse permet de conclure qu'un traitement intensif (c'est à dire court mais avec un rythme important de sessions par semaine) augmenterait l'efficacité du \gls{nfb}. A l'inverse,  
intégrer une phase de transfert durant la session diminuerait son efficacité. Par ailleurs, comme attendu, les évaluations des enseignants ne sont pas en faveur de l'efficacité du \gls{nfb}, 
ce qui est en accord avec les résultats des méta-analyses. Cette différence entre ces deux types d'évaluateurs 
peut, à première vue, s'expliquer par l'effet placebo. Cependant, il semblerait plutôt que les enseignants détecteraient moins de symptômes chez les enfants, ce qui remettrait en question la
pertinence de les utiliser pour quantifier l'effet placebo. Ici aussi, la mise à jour de ces résultats est nécessaire : plus la \gls{saob} sera appliquée sur un nombre important d'observations,
plus ses résultats seront fiables

La \gls{saob} a permis d'étudier l'impact de différents facteurs sur la performance du \gls{nfb}, cependant la personnalisation des protocoles d'entrainement dont l'influence serait intéressante 
à explorer, a été exclu de cette analyse, faute d'un nombre suffisant d'études la proposant. Ainsi, la pertinence de la personnalisation du \gls{nfb} a été étudiée ici grâce à l'analyse de la distribution 
du \gls{tbr} chez une large population d'enfants \gls{tdah}. Trois méthodes de partionnement se sont accordées sur le fait que cette distribution est bimodale, ce qui va dans 
le sens d'une personnalisation de protocoles basée la valeur de \gls{tbr} dont la valeur seuil conduisant au meilleur équilibre entre \gls{fpr} et \gls{tpr} est de 4.1. 

Ainsi, le travail présenté dans ce manuscrit a permis de décrire avec précision comment l'efficacité du \gls{nfb} est évaluée à l'aide de la technique de la méta-analyse et de donner des 
directions pour choisir les paramètres permettant d'obtenir une meilleure performance du \gls{nfb}.


\section{Perspectives}

Les résultats présentés ici, notamment ceux de la méta-analyse et de la \gls{saob}, ont vocation à être mis à jour afin d'inclure davantage d'études : les résultats alors obtenus se 
stabiliseront et leur fiabilité augmentera.   

Jusqu'à maintenant, les méta-analyses sont des outils issus des statistiques fréquentistes,
évaluant uniquement le degré de compatibilité entre les données et une hypothèse nulle.
Les courbes d'évolution des tailles d'effets et des $p$-values nous invitent à basculer vers des méta-analyses bayésiennes,
attribuant une probabilité a priori aux hypothèses testées (en utilisant l'état de l'art) et permettant ainsi d'estimer la probabilité a posteriori que l'hypothèse nulle soit vraie.
Une telle approche a été déjà été décrite et implémentée \citep{Dormuth2016, Spiegelhalter2004} mais n'a pas encore été utilisée dans le cas du \gls{nfb} appliqué au \gls{tdah}.  

Par ailleurs, de nouveaux facteurs pourront être étudiés avec la \gls{saob}, apportant ainsi de nouvelles  
indications. Dans sa forme actuelle, la \gls{saob} donne des indications qualitatives sur comment choisir les facteurs, il serait peut-être intéressant de mettre en place des
méthodes conduisant à des directives qualitatives.

Les conclusions de l'analyse de la distribution des valeurs de \gls{tbr} issues d'une large population d'enfants \gls{tdah} justifierait le recours à la 
personnalisation des protocoles de \gls{nfb} selon le profil \gls{eeg} de l'enfant. Ce genre d'approche est de plus en plus utilisée, ainsi davantage d'analyses s'intéressant
à la pertinence et à l'efficacité d'une telle approche vont être menées : il serait notamment important de déterminer si l'âge des enfants a effectivement influencé l'étude
présentée dans ce manuscrit. Par ailleurs, appliquer ces analyses sur des valeurs de \gls{tbr} obtenues les yeux fermés serait à envisager. 

Dans ce manuscrit, le \gls{nfb} a été étudié pour les enfants souffrant du \gls{tdah} mais les analyses effectuées ici pourraient être 
transposées à une toute autre application comme celles listées en \ref{NFB_applications}.
En particulier, la \gls{saob} pourrait être adaptée, c'est à dire ne pas étudier les facteurs propres au \gls{nfb} 
appliqué au \gls{tdah}, de façon à inclure toutes les études sur l'efficacité du \gls{nfb} 
quel que soit le trouble à traiter, pour déterminer si des facteurs généraux sont identifiés comme sugéré en \ref{conclusion_saob}. 

Enfin, certaines questions majeures concernant le \gls{nfb} n'ont pas été traitées dans cette thèse,
à l'instar de la définition et de la mise en place d'un protocole \textit{sham}-\gls{nfb}, déployable au sein d'un essai clinique en double aveugle. Un tel protocole 
a pour but de contrôler pour un éventuel effet placebo, effet qui est au coeur du débat sur l'efficacité du \gls{nfb} appliqué aux enfants \gls{tdah} et dont l'impact
a été discuté en \ref{pblind_anaysis}. 

Le recours au \textit{sham}-\gls{nfb} est encore assez peu répandu, toutefois des études utilisent ce type de contrôle comme \citet{Schabus2017}
qui étudient l'efficacité du \gls{nfb} sur les troubles du sommeil. Le \textit{feedback} délivré aux sujets du groupe \textit{sham}-\gls{nfb} de cette étude est calculé 
à partir des bandes de fréquence autres que celle entrainée dans le groupe \gls{nfb}. Si un tel groupe contrôle permet de conduire un essai clinique sur le \gls{nfb}
en double aveugle, on ignore cependant si ce protocole \textit{sham} n'aurait pas un impact malgré tout. Dans certaines études, le protocole le \textit{sham} retourne \textit{feedback} 
aléatoire qui ne dépend donc pas de l'activité cérébrale du sujet \citep{Arnold2014}, mais cette pratique pourrait également avoir un impact qui pour l'heure n'est pas connu.

De nombreuses pistes existent concernant l'implémentation d'un \textit{sham}-\gls{nfb}, comme par exemple garder un contrôle en temps réel des artefacts contaminant l'\gls{eeg} en 
ayant recours aux techniques présentées en \ref{steps_NFB_taining} tout 
en calculant la métrique de \textit{feedback} sur du bruit blanc. Cependant, peu de travaux ont été réalisés en la matière malgré le potentiel que représente 
un tel protocole. 

