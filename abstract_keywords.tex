% Abstract 4000 words including spaces
% keywords

% Put the english title

\begin{center}
\MakeUppercase{\LARGE{A}\Large{bstract}} \\
\vspace{0mm}
\noindent\rule{16cm}{0.4pt}
\end{center}

\textbf{Title:} Statistical analysis of electroencephalographic Neurofeedback applied to Attention Deficit Hyperactivity Disorder

% une phrase d'accroche
This thesis deals with the \gls{eegic}-\glsfirst{nfb} applied to children with \gls{adhd} whose efficacy and factors for 
improvement are studied thanks to clinical data and \gls{eegic} signals.

% définitions
\gls{eegic}-\gls{nfb} is a learning technique with a therapeutic purpose that enables to modify the \gls{eegic} activity 
as part of a conditioning paradigm that presents auditive and/or visual rewards via a serious gaming interface.
This technique has a lot of applications. This work specifically focuses on its therapeutic use for children with \gls{adhd}.
This neurodevelopmental disorder is characterized by an attention deficit disorder and/or a motor hyperactivity that
negatively impact the well being of children. The medication treatment is commonly prescribed because of its efficacy but
it can cause side effects, that's why some parents and child psychiatrists choose non-medication approaches such as \gls{eegic}-\gls{nfb}, simply noted 
\gls{nfb}. 

% problématique
\gls{nfb}, as a treatment for children with \gls{adhd}, bas been subject to various studies to prove its efficacy. 
However, they have not yet reached a consensus, which could be explained by the small number of included patients, but also by
the clinical, technical and methodological discrepancies of the studies.  

% methodes
Thus, the first purpose of this thesis is to assess the level of evidence of the \gls{nfb} efficacy for children with \gls{adhd}
through the replication and the update of a meta-analysis on this topic. 
The replication aims at settling some debates surrounding some methodological choices of the author. 
The update of the meta-analysis enables to include more studies that, despite the assumption regarding their homogeneity, 
differ from each other in terms of population and methods.

Then, this heterogeneity is studied in order to identify methodological, clinical, and technical factors that would influence
the performance of this treatment. This step is carried out thanks to three multivariate regression methods that link \gls{nfb} 
efficacy to the factors of interest. Results of each method are combined in order to determine which parameter among those
studied would indeed have an impact. 

However, some innovative factors cannot be included in this analysis because too few studies report on their performance.
Thus, the last step of this work is to analyze the relevance of personalizing the \gls{nfb} treatment based on children \gls{eegic} profil
thanks to clustering methods applied to an \gls{eegic} marker commonly linked with \gls{adhd}.

% résultats
The replication and update of the meta-analysis confirm previous results: raters who are not blind to the treatment
followed by the child observe a significant decrease of symptoms contrary to blind raters for reasons that may be more complex 
than it appears. This result is confirmed by the analysis of factors that could influence the \gls{nfb}, which also identifies 
treatment intensity. Eventually, personalizing \gls{nfb} treatment seems relevant since the clustering analysis finds two groups
of \gls{adhd} children with distinct \gls{eegic} phenotypes. 

% conclusion
To conclude, this work contributes to bring answers regarding \gls{nfb} efficacy by replicating and updating a meta-analysis as well 
as studying blind raters' assessments. Furthermore, this works enabled to determine parameters whose definition or presence may improve
\gls{nfb} efficacy.

\large{\textbf{Keywords}} : \glsfirst{nfb}, \glsfirst{eegy}, \glsfirst{adhd}, meta-analysis, bias analysis, clustering.